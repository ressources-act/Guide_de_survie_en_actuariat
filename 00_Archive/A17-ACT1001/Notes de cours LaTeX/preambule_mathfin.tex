% Pr�ambule

 % une des 2 commandes permet d'avoir des accents
	%
	\usepackage{babel}
	\usepackage[T1]{fontenc}
	  \usepackage[utf8]{inputenc}
		% permet d'inclure les r�glages pour le fran�ais
	\usepackage{amsmath,amsthm,amssymb,latexsym,amsfonts}
								%SYMBOLES MATH�MATIQUES
	\usepackage{eqnarray}	% pour faire des �quations align�s
	\usepackage{empheq}	% permet de mettre en encadr� des �quations math�matiques!
	\usepackage{bookman}	% POLICES SP�CIALES
	\usepackage{multicol}	% pour faire des colonnes multiples
%!!!!!!!! pour mettre des formules en vert standardis�e :  !!!!!!!
% \begin{empheq} [box=\fcolorbox{black}{green}]{equation} et on conclut avec \end{empheq}
	
	\usepackage{numprint}	% permet d'utiliser du contenu num�rique avanc�
	\usepackage{graphicx}	% permet d'inclure des images dans un fichier
	\usepackage{hyperref}	% permet de faire des hyperliens dans le document
	\usepackage{color,soul}	% permet de surligner en jaune	
	\usepackage[dvipsnames]{xcolor}	% permet de mettre des bo�te en couleur
	\usepackage{pdfpages}	% permet d'ins�rer un PDF dans ce fichier PDF (avec la fonction \includepdf[pages = {}]								{nomdupdf.pdf}
	\usepackage{pict2e}		% ajoute des fonctionnalit�s pour dessiner en programmation
	\usepackage[tikz]{bclogo}
	
% PACKAGE DE VINCENT GOULET ET DAVID BEAUCHEMIN POUR DES SYMBOLES D'ACTUARIAT : 
	\usepackage{actuarialangle}
	\usepackage{actuarialsymbol}
	\usepackage{cancel}
	\usepackage{colortbl}




	
% COULEUR
	\definecolor{orange}{rgb}{0.99,0.69,0.07}
	\newcommand{\orange}{\textcolor{orange}}
	\newcommand{\red}{\textcolor{red}}
	\newcommand{\blue}{\textcolor{blue}}
	\newcommand{\green}{\textcolor{green}}
	\newcommand{\purple}{\textcolor{magenta}}
	\newcommand{\yellow}{\textcolor{yellow}}	
	\newcommand{\darkgreen}{\textcolor{Green}}
	\newcommand{\navyblue}{\textcolor{NavyBlue}}


% OPTIONS AJOUT�ES POUR LES COULEURS DE TOUCHE DE LA CALCULATRICE	

	\newcommand{\touche}{\fcolorbox{black}{lightgray}}
	\newcommand{\second}{\fcolorbox{black}{yellow}{2nd}}
	\newcommand{\eval}{\biggr\rvert}
	
% CR�ATION DE NOUVELLES COMMANDES, pour se faire des raccourcis
	\newcommand{\p}{\paragraph{}}
	\newcommand{\egal}{\Leftrightarrow}
	\newcommand{\n}{\newline{}}
	
% Cr�ation de �Theorem� : �a me cr�e une s�rie de th�or�me, convention, exemple num�rot�s
	\newtheorem{conv}{Convention}
	\newtheorem{definition}{\purple{D�finition}}
	\newtheorem{remarque}{\blue{Remarque}}	
	
% d�finition des symboles math�matiques : entiers, reels
	\newcommand{\reels}{\mathbb{R}}
	\newcommand{\entiers}{\mathbb{Z}}
	
% cr�ation d'un nouvel environnement pour les formules math�matiques encadr�es en vert qui reviennent tr�s souvent!


% package enuitem pour personnaliser ses listes  � puces et num�ro!
	\usepackage{enumitem}	


% Pr�sentation document
	\title{Notes de cours \\ Math�matiques financi�res \\ Automne 2017 \\ ACT-1001}
	\author{Gabriel Cr�peault-Cauchon}
	\date{Derni�re mise � jour : \today}

% fin du pr�ambule	
