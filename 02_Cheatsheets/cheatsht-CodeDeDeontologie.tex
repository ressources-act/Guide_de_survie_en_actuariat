\documentclass[10pt, french]{article}
\usepackage[frenchb]{babel}

%% -----------------------------
%% Préambule
%% -----------------------------
% !TEX encoding = UTF-8 Unicode
% LaTeX Preamble for all cheatsheets
% Author : Gabriel Crépeault-Cauchon

% HOW-TO : copy-paste this file in the same directory as your .tex file, and add in your preamble the next command right after you have specified your documentclass : 
% \input{preamble-cheatsht.tex}
% ---------------------------------------------
% ---------------------------------------------

% Extra note : this preamble creates document that are meant to be used inside the multicols environment. See the documentation on internet for further information.

%% -----------------------------
%% Encoding packages
%% -----------------------------
\usepackage[utf8]{inputenc}
\usepackage[T1]{fontenc}
\usepackage{babel}
\usepackage{lmodern}
\usepackage[colorinlistoftodos]{todonotes}
%% -----------------------------
%% Variable definition
%% -----------------------------
\def\auteur{\href{https://github.com/ressources-act/Guide_de_survie_en_actuariat/blob/master/02_Cheatsheets/contributeurs/contributeurs-cheatshts.pdf}{\faGithub \ Liste des contributeurs}}
\def\BackgroundColor{white}
\usepackage{xargs} % for more logical new function creation

%% -----------------------------
%% Margin and layout
%% -----------------------------
% Determine the margin for cheatsheet
\usepackage[landscape, hmargin=1cm, vmargin=1.7cm]{geometry}
\usepackage{multicol}

% Remove automatic indentation after section/subsection title.
\setlength{\parindent}{0cm}

% Save space in cheatsheet by removing space between align environment and normal text.
\usepackage{etoolbox}
\newcommand{\zerodisplayskips}{%
  \setlength{\abovedisplayskip}{0pt}%
  \setlength{\belowdisplayskip}{0pt}%
  \setlength{\abovedisplayshortskip}{0pt}%
  \setlength{\belowdisplayshortskip}{0pt}}
\appto{\normalsize}{\zerodisplayskips}
\appto{\small}{\zerodisplayskips}
\appto{\footnotesize}{\zerodisplayskips}

%% -----------------------------
%% URL and links
%% -----------------------------
\usepackage{hyperref}
\hypersetup{colorlinks = true, urlcolor = gray!70!white, linkcolor = black}

%% -----------------------------
%% Document policy (uncomment only one)
%% -----------------------------
%	\usepackage{concrete}
	\usepackage{mathpazo}
%	\usepackage{frcursive} %% permet d'écrire en lettres attachées
%	\usepackage{aeguill}
%	\usepackage{mathptmx}
%	\usepackage{fourier} 

%% -----------------------------
%% Math configuration
%% -----------------------------
\usepackage[fleqn]{amsmath}
\usepackage{amsthm,amssymb,latexsym,amsfonts}
\usepackage{gensymb}
\usepackage{empheq}
\usepackage{numprint}
\usepackage{dsfont} % Pour avoir le symbole du domaine Z
%\usepackage{bigints} % pour des gros intégrales
% Mathematics shortcuts
\usepackage{scalerel,stackengine,amsmath}
\newcommand\equalhat{\mathrel{\stackon[1.5pt]{=}{\stretchto{%
    \scalerel*[\widthof{=}]{\wedge}{\rule{1ex}{3ex}}}{0.5ex}}}}
\newcommand{\reels}{\mathbb{R}}
\newcommand{\entiers}{\mathbb{Z}}
\newcommand{\naturels}{\mathbb{N}}
\newcommand{\eval}{\biggr \rvert}
\usepackage{cancel}
\newcommand{\derivee}[1]{\frac{\partial}{\partial #1}}
\newcommand{\prob}[1]{\Pr \left( #1 \right)}
\newcommand{\esp}[1]{\mathrm{E} \left[ #1 \right]} % espérance
\newcommand{\variance}[1]{\mathrm{Var} \left( #1   \right)}
\newcommand{\covar}[1]{\mathrm{Cov} \left( #1   \right)}
\newcommand{\laplace}{\mathcal{L}}
\newcommand{\deriv}[3][]{\frac{\partial^{#1}#3}{\partial #2^{#1}}}
\newcommand{\e}[1]{\mathrm{e}^{#1}}
\newcommand{\te}[1]{\text{exp}\left\{#1\right\}}
\DeclareMathSymbol{\shortminus}{\mathbin}{AMSa}{"39}
%%	Example usage:	\sumz{n}{i = 1} <=> \overset{n}{\underset{i = 1}{\sum}}
\newcommand{\sumz}[2]{\overset{#1}{\underset{#2}{\sum}}}
%%	Example usage:	\limz{h}{0} <=> \underset{h \rightarrow 0}{\lim}
\newcommand{\limz}[2]{\underset{#1 \rightarrow #2}{\lim}}
%%	Example usage:	\LVx{h}	<=>	\actsymb[h]{L}{}[]
%%					\LVx[n]{h}	<=>	\actsymb[h]{L}{}[n]
\newcommand{\LVx}[2][]{\actsymb[#2]{L}{}[#1]}
\DeclareMathOperator*{\argmax}{arg\,max}
\DeclareMathOperator*{\argmin}{arg\,min}
%%%	\icbox{<frame color>}{<background color>}{<text>}
\newcommandx{\icbox}[3][1 = bleudefrance, 2 = beaublue]{\fcolorbox{#1}{#2}{#3}}
%%	other good color combo is azure(colorwheel) arsenic
\usepackage{longfbox}
%	voir cette page, paquetage avec CSS https://ctan.math.illinois.edu/macros/latex/contrib/longfbox/longfbox.html
\newfboxstyle{rappel}{
	background-color = tealblue!20!white, 
	border-style = outset,
	breakable = true,
%	
	border-color = tealblue,
	border-radius = 1ex, 
%
	padding-bottom = 0.2ex,
	padding-top = 0.2ex,
	padding-left = 0.4ex,
	padding-right = 0.4ex,
%	
	border-top-width = 0.3ex,
	border-bottom-width = 0.3ex,
%
	border-left-width = 1ex, 
	border-bottom-left-radius = 0.2ex,
%	
	border-right-width = 1ex, 
	border-top-right-radius = 0.2ex,
%	
}
\newfboxstyle{formula}{ 
	background-color = beaublue, 
	border-color = bleudefrance
}
\newfboxstyle{imphl}{ 
	padding = 0pt,
	margin = 0pt,
	baseline-skip = false,
	background-color = palechestnut!60!white, 
	border-color = white
}
\newfboxstyle{conditions}{ 
	background-color = palechestnut, 
	border-color = red
}
\newcommandx{\rcbox}[3][1 = bleudefrance, 2 = beaublue]{\lfbox[border-radius = 0.5ex, background-color = #2, border-color = #1]{#3}}

% To indicate equation number on a specific line in align environment
\newcommand\numberthis{\addtocounter{equation}{1}\tag{\theequation}}

%
% Actuarial notation packages
%
\usepackage{actuarialsymbol}
\usepackage{actuarialangle}

%
% Matrix notation for math symbols (\bm{•})
%
\usepackage{bm}
% Matrix notation variable (bold style)
\newcommand{\matr}[1]{\mathbf{#1}}



%% -----------------------------
%% tcolorbox configuration
%% -----------------------------
\usepackage[most]{tcolorbox}
\tcbuselibrary{xparse}
\tcbuselibrary{breakable}

%%
%% Coloured box "definition" for definitions
%%
\DeclareTColorBox{definition}{ o }				% #1 parameter
{
	colframe=black,colback=white, % color of the box
	breakable, 
	pad at break* = 0mm, 						% to split the box
	title = {#1},
	after title = {\large \hfill \faBook},
}
%%
%% Coloured box "definition2" for definitions
%%
\DeclareTColorBox{definitionNOHFILL}{ o }				% #1 parameter
{
	colframe=blue!60!green,colback=blue!5!white, % color of the box
	pad at break* = 0mm, 						% to split the box
	title = {#1},
	before title = {\faBook \quad },
	breakable
}
%%
%% Coloured box "definition2" for definitions
%%
\DeclareTColorBox{definitionNOHFILLsub}{ o }				% #1 parameter
{
	colframe=blue!40!green,colback=blue!5!white, % color of the box
	pad at break* = 0mm, 						% to split the box
	title = {#1},
	before title = {\faNavicon \quad }, %faBars  faGetPocket
	breakable
}
%%
%% Coloured box "definition3" for propriétés
%%
\DeclareTColorBox{definitionNOHFILLprop}{ o }				% #1 parameter
{
	colframe=amber(sae/ece),colback=amber(sae/ece)!5!white, % color of the box
	pad at break* = 0mm, 						% to split the box
	title = {#1},
	before title = {\faGetPocket \quad }, %faBars  faGetPocket
	breakable
}
%%
%% Coloured box "definition3" for propriétés
%%
\DeclareTColorBox{definitionNOHFILLpropos}{ o }				% #1 parameter
{
	colframe=carmine,colback=carmine!5!white, % color of the box
	pad at break* = 0mm, 						% to split the box
	title = {#1},
	before title = {\faColumns \quad }, %\faEllipsisH  faColumns
	breakable
}


%%
%% Coloured box "algo" for algorithms
%%
\newtcolorbox{algo}[ 1 ]
{
	colback = blue!5!white,
	colframe = blue!75!black,
	title=#1,
	fonttitle = \bfseries,
	breakable
}
%%
%% Coloured box "conceptgen" for points adding to a concept's deifintion
%%
\newtcolorbox{conceptgen}[ 1 ]
{
	breakable,
	colback = beaublue,
	colframe = airforceblue,
	title=#1,
	fonttitle = \bfseries
}
%%
%% Coloured box "rappel" pour rappel de formules
%%
\DeclareTColorBox{conceptgen_enhanced}{ o }
{
	enhanced,
	title = #1,
	colback=beaublue, % color of the box
%	colframe=blue(pigment),
%	colframe=arsenic,	
	colbacktitle=airforceblue,
	fonttitle = \bfseries,
	breakable,
	boxed title style={size=small,colframe=arsenic} ,
	attach boxed title to top center = {yshift=-3mm,yshifttext=-1mm},
}
%%
%% Coloured box "probch1" pour formules relatives au 1er chapitre de prob
%%
\newtcolorbox{probch1}[ 1 ]
{
	colback = ao(english)!40!white,
	colframe = forestgreen(traditional),
	fonttitle = \bfseries,	
	breakable,
	title=#1
}
%%
%% Coloured box "probch2" pour formules relatives au 2e chapitre de prob
%%
\newtcolorbox{probch2}[ 1 ]
{
	colback = orange!50!white,
	colframe = burntorange,
	fonttitle = \bfseries,	
	breakable,
	title=#1
}
%%
%% Coloured box "axioms" pour formules relatives à la dernière partie du chapitre 2 de prob
%%
\newtcolorbox{axioms}[ 1 ]
{
	colback = blue!10!white,
	colframe = blue!70!white,
	fonttitle = \bfseries,	
	breakable,
	title=#1
}
%%
%% Coloured box "probch3" pour formules relatives au 3ème chapitre de prob
%%
\newtcolorbox{probch3}[ 1 ]
{
	colback = ruddypink,
	colframe = burgundy,
	fonttitle = \bfseries,	
	breakable,
	title=#1
}
%%
%% Coloured box "formula" for formulas
%%
\newtcolorbox{formula}[ 1 ]
{
	colback = green!5!white,
	colframe = green!70!black,
	breakable,
	fonttitle = \bfseries,
	title=#1
}
%%
%% Coloured box "formula" for formulas
%%
\DeclareTColorBox{algo2}{ o }
{
	enhanced,
	title = #1,
	colback=blue!5!white,	
	colbacktitle=blue!75!black,
	fonttitle = \bfseries,
	breakable,
	boxed title style={size=small,colframe=arsenic} ,
	attach boxed title to top center = {yshift=-3mm,yshifttext=-1mm},
}
%%
%% Coloured box "examplebox" for formulas
%%
\newtcolorbox{examplebox}[ 1 ]
{
	colback = beaublue,
	colframe = amethyst,
	breakable,
	fonttitle = \bfseries,title=#1
}
%%
%% Coloured box "rappel" pour rappel de formules
%%
\newtcolorbox{rappel}[ 1 ]
{
	colback = ashgrey,
	colframe = arsenic,
	breakable,
	fonttitle = \bfseries,title=#1
}
%%
%% Coloured box "rappel" pour rappel de formules
%%
\DeclareTColorBox{rappel_enhanced}{ o }
{
	enhanced,
	title = #1,
	colback=ashgrey, % color of the box
%	colframe=blue(pigment),
%	colframe=arsenic,	
	colbacktitle=arsenic,
	fonttitle = \bfseries,
	breakable,
	boxed title style={size=small,colframe=arsenic} ,
	attach boxed title to top center = {yshift=-3mm,yshifttext=-1mm},
}
%%
%% Coloured box "notation" for notation and terminology
%%
\DeclareTColorBox{distributions}{ o }			% #1 parameter
{
	enhanced,
	title = #1,
	colback=gray(x11gray), % color of the box
%	colframe=blue(pigment),
	colframe=arsenic,	
	colbacktitle=aurometalsaurus,
	fonttitle = \bfseries,
	boxed title style={size=small,colframe=arsenic} ,
	attach boxed title to top center = {yshift=-3mm,yshifttext=-1mm},
	breakable
%	left=0pt,
%  	right=0pt,
%    box align=center,
%    ams align*
%  	top=-10pt
}
\newtcolorbox{contrib}[ 1 ]
{
	colback = babyblueeyes,
	colframe = airforceblue,
	fonttitle = \bfseries,
	title = {#1},
	valign = center
}

%% -----------------------------
%% Graphics and pictures
%% -----------------------------
\usepackage{graphicx}
\usepackage{pict2e}
\usepackage{tikz}

%% -----------------------------
%% insert pdf pages into document
%% -----------------------------
\usepackage{pdfpages}

%% -----------------------------
%% Color configuration
%% -----------------------------
\usepackage{color, soulutf8, colortbl}


%
%	Colour definitions
%
\definecolor{armygreen}{rgb}{0.29, 0.33, 0.13}	%	army
\definecolor{asparagus}{rgb}{0.53, 0.66, 0.42}	% pastel green militariesque
\definecolor{britishracinggreen}{rgb}{0.0, 0.26, 0.15}
\definecolor{calpolypomonagreen}{rgb}{0.12, 0.3, 0.17}
\definecolor{darkgreen}{rgb}{0.0, 0.2, 0.13}
\definecolor{lightgreen}{rgb}{0.2, 0.95, 0.2}

\definecolor{antiquebrass}{rgb}{0.8, 0.58, 0.46}	% brown-ish light cardboard color

\definecolor{blue(munsell)}{rgb}{0.0, 0.5, 0.69}
\definecolor{blue(matcha)}{rgb}{0.596, 0.819, 1.00}
\definecolor{blue(munsell)-light}{rgb}{0.5, 0.8, 0.9}
\definecolor{bleudefrance}{rgb}{0.19, 0.55, 0.91}
\definecolor{blizzardblue}{rgb}{0.67, 0.9, 0.93}	%	mr.freeze light baby blue 
\definecolor{bondiblue}{rgb}{0.0, 0.58, 0.71}	%	darker cyan type inidgo blue
\definecolor{blue(pigment)}{rgb}{0.2, 0.2, 0.6}
\definecolor{bluebell}{rgb}{0.64, 0.64, 0.82}
\definecolor{airforceblue}{rgb}{0.36, 0.54, 0.66}
\definecolor{beaublue}{rgb}{0.74, 0.83, 0.9}    % almost white
\definecolor{blue_rectangle}{RGB}{83, 84, 244}		% ACT-2004
\definecolor{cobalt}{rgb}{0.0, 0.28, 0.67}	% nice light blue-ish
\definecolor{ballblue}{rgb}{0.13, 0.67, 0.8}	%	almost green ish blue ish
\definecolor{babyblueeyes}{rgb}{0.63, 0.79, 0.95}

\definecolor{indigo(web)}{rgb}{0.29, 0.0, 0.51}	% purple-ish
\definecolor{antiquefuchsia}{rgb}{0.57, 0.36, 0.51}	%	pastel matte (darkerish) purple ish
\definecolor{darkpastelpurple}{rgb}{0.59, 0.44, 0.84}	%	pretty purple
\definecolor{gray(x11gray)}{rgb}{0.75, 0.75, 0.75}
\definecolor{aurometalsaurus}{rgb}{0.43, 0.5, 0.5}
\definecolor{bulgarianrose}{rgb}{0.28, 0.02, 0.03}	%	dark maroon type 
\definecolor{pastelred}{rgb}{1.0, 0.41, 0.38}		%	light red pinktinybit ish
\definecolor{lightmauve}{rgb}{0.86, 0.82, 1.0}
\definecolor{eggshell}{rgb}{0.94, 0.92, 0.84}
\definecolor{azure(colorwheel)}{rgb}{0.0, 0.5, 1.0}
\definecolor{darkgreen}{rgb}{0.0, 0.2, 0.13}			
\definecolor{ao(english)}{rgb}{0.0, 0.5, 0.0}		% prertty apple dark pastel (light) green
\definecolor{green_rectangle}{RGB}{131, 176, 84}		% ACT-2004
\definecolor{red_rectangle}{RGB}{241,112,113}		% ACT-2004
\definecolor{amethyst}{rgb}{0.6, 0.4, 0.8}
\definecolor{amethyst-light}{rgb}{0.6, 0.4, 0.8}
\definecolor{ruddypink}{rgb}{0.88, 0.56, 0.59}

\definecolor{amber(sae/ece)}{rgb}{1.0, 0.49, 0.0} 	%	pretty orange ish
\definecolor{burntsienna}{rgb}{0.91, 0.45, 0.32}		%%	lighter pastel orange
\definecolor{burntorange}{rgb}{0.8, 0.33, 0.0}		%%	similar but deeper orange
\definecolor{orange-red}{rgb}{1.0, 0.27, 0.0}

\definecolor{tealblue}{rgb}{0.21, 0.46, 0.53}

\definecolor{battleshipgrey}{rgb}{0.52, 0.52, 0.51}  % lilght ish gray
\definecolor{ashgrey}{rgb}{0.7, 0.75, 0.71}			% dark grey-black-ish
\definecolor{arsenic}{rgb}{0.23, 0.27, 0.29}			% light green-beige-ish gray
\definecolor{gray(x11gray)}{rgb}{0.75, 0.75, 0.75}

\definecolor{carmine}{rgb}{0.59, 0.0, 0.09} 			% deep red
\definecolor{amaranth}{rgb}{0.9, 0.17, 0.31}
\definecolor{brickred}{rgb}{0.8, 0.25, 0.33}
\definecolor{chestnut}{rgb}{0.8, 0.36, 0.36}		% pink red ish light
\definecolor{palechestnut}{rgb}{0.87, 0.68, 0.69}
\definecolor{pastelred}{rgb}{1.0, 0.41, 0.38}
\definecolor{forestgreen(traditional)}{rgb}{0.0, 0.27, 0.13}
%
% Useful shortcuts for coloured text
%
\newcommand{\orange}{\textcolor{orange}}
\newcommand{\red}{\textcolor{red}}
\newcommand{\cyan}{\textcolor{cyan}}
\newcommand{\blue}{\textcolor{blue}}
\newcommand{\green}{\textcolor{green}}
\newcommand{\purple}{\textcolor{magenta}}
\newcommand{\yellow}{\textcolor{yellow}}

%% -----------------------------
%% Enumerate environment configuration
%% -----------------------------
%
% Custum enumerate & itemize Package
%
\usepackage{enumitem}
%
% French Setup for itemize function
%
\frenchbsetup{StandardItemLabels=true}
%
% Change default label for itemize
%
\renewcommand{\labelitemi}{\faAngleRight}


%% -----------------------------
%% Tabular column type configuration
%% -----------------------------
\newcolumntype{C}{>{$}c<{$}} % math-mode version of "l" column type
\newcolumntype{L}{>{$}l<{$}} % math-mode version of "l" column type
\newcolumntype{R}{>{$}r<{$}} % math-mode version of "l" column type
\newcolumntype{f}{>{\columncolor{green!20!white}}p{1cm}}
\newcolumntype{g}{>{\columncolor{green!40!white}}m{1.2cm}}
\newcolumntype{a}{>{\columncolor{red!20!white}$}p{2cm}<{$}}	% ACT-2005
% configuration to force a line break within a single cell
\usepackage{makecell}


%% -----------------------------
%% Fontawesome for special symbols
%% -----------------------------
\usepackage{fontawesome}

%% -----------------------------
%% Section Font customization
%% -----------------------------
\usepackage{sectsty}
\sectionfont{\color{\SectionColor}}
\subsectionfont{\color{\SubSectionColor}}
\subsubsectionfont{\color{\SubSubSectionColor}}

%% -----------------------------
%% Footer/Header Customization
%% -----------------------------
\usepackage{lastpage}
\usepackage{fancyhdr}
\pagestyle{fancy}

%
% Header
%
\fancyhead{} 	% Reset
\fancyhead[L]{Aide-mémoire pour~ \cours ~(\textbf{\sigle})}
\fancyhead[R]{\auteur}

%
% Footer
%
\fancyfoot{}		% Reset
\fancyfoot[R]{\thepage ~de~ \pageref{LastPage}}
\fancyfoot[L]{\href{https://github.com/ressources-act/Guide_de_survie_en_actuariat}{\faGithub \ ressources-act/Guide de survie en actuariat}}
%
% Page background color
%
\pagecolor{\BackgroundColor}




%% END OF PREAMBLE
% ---------------------------------------------
% ---------------------------------------------
%% -----------------------------
%% Redefine from template
%% -----------------------------
\def\auteur{Alec James van Rassel}
%% -----------------------------
%% Variable definition
%% -----------------------------
\def\cours{Législation et responsabilité professionnelle}
\def\sigle{ACT-3001}
%% -----------------------------
%% Colour setup for sections
%% -----------------------------
\def\SectionColor{burntorange}
\def\SubSectionColor{burntsienna}
\def\SubSubSection{burntsienna}

%% Reduce margin space
\setlength{\abovedisplayskip}{-15pt}
\setlist{leftmargin=*}
\setcounter{secnumdepth}{1}

% 
% Débute numérotation des chapitres à 2 pour suivre les notes de Marie-Piere.
% 
\setcounter{section}{1}

%% -----------------------------
%% Début du document
%% -----------------------------
\begin{document}


\begin{center}
	\textsc{\Large Contributeurs}\\[0.5cm] 
\end{center}
\begin{contrib}{Code de Déontologie}
\begin{description}
	\item[aut., cre.] Alec James van Rassel
\end{description}

\textbf{\underline{Référence}}
\begin{description}
	\item[src.] Institut Canadien des Actuaire
	\item[src.] Michel Simard, présentation
\end{description}

\textbf{\underline{Contributeurs}}
\begin{description}
	\item[pfr.]	QQ1
\end{description}
\end{contrib}

\newpage

\begin{multicols*}{2} 
\setcounter{secnumdepth}{0}
\section{Code de déontologie de l'ICA}
\begin{definitionGENERAL}{Intégrité professionnelle}[\circled{1}{black}][azure(colorwheel)]
Le membre agit avec : 
\begin{itemize}
	\item	honnêteté, 
	\item	intégrité et compétence, 
	\item	et de manière à : 
		\begin{enumerate}
		\item	remplir les responsabilités de la profession envers le public et 
		\item	à maintenir la réputation de la profession actuarielle.
		\end{enumerate}
\end{itemize}

\bigskip

\begin{rappel_enhanced}[Exemples de non-respect]
Par exemple, des erreurs significatives dans des évaluations de régimes de retraite n’ont pas été divulguées au client en temps opportun. Les membres qui ne collaborent pas avec d’autres membres dans l’intérêt du client : nous voyons parfois cette situation lorsque le client fait appel à un nouveau conseiller et que le membre subissant une perte d’entreprise ne collabore pas en temps opportun.

\begin{itemize}
	\item	Cacher des erreurs d'un client.
	\item	Ne pas collaborer avec d'autre membres dans l'intérêt du client.
	\begin{itemize}
		\item	Par exemple, lorsque le client fait appel à un nouveau conseiller et que le membre subissant une perte d’entreprise ne collabore pas en temps opportun.
	\end{itemize}
\end{itemize}
\end{rappel_enhanced}
\end{definitionGENERAL}

\begin{definitionGENERAL}{Normes de qualification}[\circled{2}{black}][azure(colorwheel)]
Le membre ne rend des \og \textit{services professionnels} \fg{} que :
\begin{enumerate}
	\item	s'il est qualifié pour le faire \textbf{et} 
	\item	qu'il satisfait aux normes de qualification applicables.
\end{enumerate}

\bigskip

\begin{rappel_enhanced}[Exemple de non-respect]
\begin{itemize}
	\item	Pratiquer à l'extérieur de son domaine d'expertise.
\end{itemize}
\end{rappel_enhanced}
\end{definitionGENERAL}

\begin{definitionNOHFILLsub}[Annotation 2-1]
Le membre a la responsabilité professionnelle de respecter les normes de qualification professionnelle continue promulguées par un organisme actuariel reconnu dans la juridiction où le membre rend ses services professionnels et de se tenir au courant des révisions apportées à ces normes.

\bigskip

Les \og normes de qualification professionnelle continue \fg{} promulguées par l’Institut canadien des actuaires sont définies à l’article 1.01(16.01.2) des statuts administratifs.

\textbf{Définition de qualification}:
\begin{itemize}
	\item	Juridiction
	\item	Branche/segment d’affaires
	\item	Type de travail (réserves par rapport à tarification)
\end{itemize}
\end{definitionNOHFILLsub}

\bigskip

\begin{definitionGENERAL}{Normes de pratique}[\circled{3}{black}][azure(colorwheel)]
Le membre veille à ce que les \og \textit{services professionnels} \fg{} rendus par lui \textit{ou} sous sa direction répondent aux normes de pratique pertinentes.

\bigskip

\begin{rappel_enhanced}[Exemple de potentiel non-respect]
\begin{itemize}
	\item	Des hypothèses agressives pour réduire les coûts.
\end{itemize}

\paragraph{Note}	La méconnaissance d'une question n'est pas une excuse.
\end{rappel_enhanced}
\end{definitionGENERAL}

\begin{definitionNOHFILLsub}[Annotation 3-1]
Le membre a la responsabilité professionnelle de respecter les normes de pratique promulguées par un organisme actuariel reconnu dans la juridiction où le membre rend ses services professionnels et de se tenir au courant des principes et normes de pratique reconnus dans la juridiction où il rend ses services professionnels.
\end{definitionNOHFILLsub}

\bigskip

\begin{definitionGENERAL}{Divulgation (rémunération)}[\circled{4}{black}][azure(colorwheel)]
Le membre divulgue entièrement et sans retard au client \textcolor{teal}{\textit{ou} à l'employeur} la source de toute rémunération indirecte \textit{ou} directe que lui \textcolor{teal}{\textit{ou} sa compagnie} ont reçue \textit{ou} peuvent recevoir relativement à une mission dans le cadre de laquelle il rend des \og \textit{services professionnels} \fg{} à ce client \textcolor{teal}{\textit{ou} à cet employeur}.
\end{definitionGENERAL}

\begin{definitionNOHFILLsub}[Annotation 4-1]
\og Divulgation entière et sans retard \fg{} désigne une divulgation de tous les faits importants relativement à la rémunération indirecte \textit{ou} directe qui peuvent être pertinents \textbf{aux fins de la décision} du client \textcolor{teal}{\textit{ou} de l'employeur} et une divulgation suffisamment rapide pour permettre au client ou \textcolor{teal}{\textit{ou} à l'employeur} de prendre une décision éclairée et indépendante. Cette divulgation devrait se faire par écrit.
\end{definitionNOHFILLsub}

\begin{definitionNOHFILLsub}[Annotation 4-2]
S’il n’est pas indépendant sur les plans financier et organisationnel relativement à toute question liée aux services professionnels rendus, le membre devrait divulguer entièrement et sans retard au client \textcolor{teal}{\textit{ou} à l'employeur} toute relation pertinente non apparente.
\end{definitionNOHFILLsub}

\begin{definitionNOHFILLsub}[Annotation 4-3]
Le membre qui travaille pour une compagnie établie à plusieurs endroits est soumis aux exigences de divulgation entière et sans retard des sources de la rémunération que la compagnie pour laquelle il travaille a reçue \textit{ou} peut recevoir en rapport avec les services professionnels qui touchent le mandat spécifique exécuté pour ce client, sans égard à l’endroit où cette rémunération est reçue.
\end{definitionNOHFILLsub}

\bigskip


\begin{definitionGENERAL}{Conflits d'intérêts}[\circled{5}{black}][azure(colorwheel)]
Le membre ne rend pas de \og \textit{services professionnels} \fg{} qui puissent susciter un conflit d'intérêt rééel ou potentiel, sauf :
\begin{enumerate}[label = (\alph*)]
	\item	si la capacité du membre d'agir équitablement n'est pas affaiblie;
	\item	si le conflit d'intérêt a été divulgué entièrement et sans retard à tous les \og \textit{utilisateurs directs} \fg{}  actuels et éventuels connus; et
	\item	si tous les \og \textit{utilisateurs directs} \fg{} actuels et éventuels connus ont expressément convenu de l'exécution des services par le membre.
\end{enumerate}

\bigskip

\begin{rappel_enhanced}[Exemple de non-respect]
\begin{itemize}
	\item	Dans le cadre d'un implication antérieure auprès d'une firme de placement, un membre a reçu des commissions indirectes. 
	\item	Le membre recommande cette firme de placement à un autre client sans divulguer le conflit d’intérêt possible. 
	\item	Il est important que les membres divulguent et documentent adéquatement tout conflit d’intérêt possible.
\end{itemize}
\end{rappel_enhanced}
\end{definitionGENERAL}

\begin{definitionNOHFILLsub}[Annotation 5-1]
\og Divulgation entière et sans retard \fg{} désigne une divulgation de tous les faits importants relativement au conflit, incluant 
\begin{enumerate}
	\item	la nature de l’influence ou de la relation, de même que 
	\item	la nature et l’ampleur de l’intérêt,
\end{enumerate}
qui peuvent être pertinents aux fins de la décision de l’utilisateur direct et une divulgation suffisamment rapide pour permettre à l’utilisateur direct de prendre une décision éclairée et indépendante. Cette divulgation devrait se faire par écrit.  
\end{definitionNOHFILLsub}

\bigskip

\begin{definitionGENERAL}{Contrôle du produit}[\circled{6}{black}][azure(colorwheel)]
Le membre qui rend des \og \textit{services professionnels} \fg{} doit prendre des mesures raisonnables pour s'assurer que ces services ne soient pas utilisés pour :
\begin{itemize}
	\item	induire en erreur d'autres parties ou 
	\item	pour enfreindre ou courner la loi.
\end{itemize}
\end{definitionGENERAL}

\begin{definitionNOHFILLsub}[Annotation 6-1]
Les documents préparés par le membre peuvent être utilisés par une autre partie d’une manière susceptible d’influer sur les agissements d’un tiers. Le membre devrait reconnaître le risque que ces documents puissent être mal cités, mal interprétés ou autrement mal utilisés et prendre des mesures raisonnables pour veiller à ce que l’information soit \textbf{précise} et \textbf{présentée de façon équitable}. 
\end{definitionNOHFILLsub}

\begin{definitionNOHFILLsub}[Annotation 6-2]
Les documents préparés par le membre devraient identifier 
\begin{itemize}
	\item	le client ou l’employeur pour lequel ils sont préparés
	\item	celui-ci comme source des renseignements
	\item	en quelle qualité le membre a agi à cet égard
	\item	tout aspect des services professionnels dont le membre n’assume pas la responsabilité et
	\item	la mesure dans laquelle le membre ou d’autres personnes seront disponibles pour fournir au client ou à l’employeur des renseignements et des explications supplémentaires le cas échéant
\end{itemize}
\end{definitionNOHFILLsub}

\bigskip

\begin{definitionGENERAL}{Confidentialité}[\circled{7}{black}][azure(colorwheel)]
Nul membre ne peut divulguer à une autre partie les \og \textit{renseignements confidentiels} \fg{} obtenus dans le cadre d'une mission professionnelle exécutée pour le compte d'un client \textcolor{teal}{\textit{ou} d'un employeur}, sauf :
\begin{itemize}
	\item	s'il y est autorisé explicitement ou implicitement par le client \textcolor{teal}{\textit{ou} l'employeur}, 
	\item	qu'il y est tenu en vertu de la Règle 13, ou
	\item	si le Conseil de déontologie, une équipe d'enquête, un tribunal disciplinaire ou un tribunal d'appel lui en fait la demande relativement à toute question disciplinaire préue à la section 20 des statuts administratifs, ou
	\item	qu'il y est tenu par la loi.
\end{itemize}
\end{definitionGENERAL}

\begin{definitionGENERAL}{Courtoisie et collaboration}[\circled{8}{black}][azure(colorwheel)]
Le membre :
\begin{enumerate}
	\item	rend ses \og \textit{services professionnels} \fg{} avec courtoisie et respect professionnel, 
	\item	évite les critiques injustifiables ou déplacées à l'égard d'autres membres et 
	\item	accorde sa collaboration aux autres dans l'intérêt du client \textcolor{teal}{\textit{ou} de l'employeur}.
\end{enumerate}
\end{definitionGENERAL}

\begin{definitionNOHFILLsub}[Annotation 8-1]
Il peut surgir des différences d’opinions entres membres, particulièrement dans le choix des hypothèses et des méthodes. Les discussions sur ces différences, que ce soit directement entre membres ou dans le cadre d’observations présentées à un client par un membre sur le travail d’un autre, devraient se \textbf{dérouler en toute objectivité ainsi qu’avec courtoisie et respect}.
\end{definitionNOHFILLsub}

\begin{definitionNOHFILLsub}[Annotation 8-2]
Dans le cours d’un mandat ou de son emploi, le membre peut se trouver dans une situation telle qu’il serait dans les meilleurs intérêts du client ou de l’employeur que le membre formule une opinion différente de celle exprimée par un autre membre, ainsi qu’une explication des facteurs qui appuient cette deuxième opinion. Aucun article des Règles ne peut être réputé empêcher le membre d’exprimer au client ou à l’employeur cette deuxième opinion.
\end{definitionNOHFILLsub}

\begin{definitionNOHFILLsub}[Annotation 8-3]
Si le membre est invité à conseiller un client ou un employeur et qu’il sait ou a des motifs raisonnables de croire qu’un autre membre agit déjà à titre professionnel relativement à la même question ou a agi en cette qualité récemment, il serait normalement prudent de consulter l’autre membre, afin de se préparer adéquatement au mandat et de formuler un jugement informé quant à savoir s’il existe des circonstances relatives à une infraction potentielle aux Règles, qui pourraient influer sur l’acceptation ou le refus du mandat.
\end{definitionNOHFILLsub}

\bigskip

\begin{definitionGENERAL}{Publicité}[\circled{9}{black}][azure(colorwheel)]
Le membre ne s'addone à quelque activité de publicité ou de sollicitation d'affaires à l'égard de \og \textit{services professionnels} \fg{} dont : 
\begin{itemize}
	\item	il sait ou devrait savoir qu'elle est fausse ou trompeuse ou 
	\item	qui donne une image défavorable de la profession ou de la compétence et de l'intégrité de l'un de ses membres.
\end{itemize}
\end{definitionGENERAL}

\begin{definitionNOHFILLsub}[Annotation 9-1]
\og Publicité et sollicitation d’affaires \fg{} englobe \textbf{toutes les communications}, quel que soit le moyen, communications verbales comprises, qui peuvent directement ou indirectement exercer une influence sur une personne ou une organisation dans la décision de savoir si 
\begin{itemize}
	\item	elle a besoin de services professionnels ou
	\item	pour choisir telle personne ou telle entreprise pour l’exécution de tels services professionnels.
\end{itemize}
\end{definitionNOHFILLsub}

\bigskip

\begin{definitionGENERAL}{Titres et désignations}[\circled{10}{black}][azure(colorwheel)]
Le membre n'utilise les titres d'affiliation et désignations d'un \og \textit{organisme actuariel recconu} \fg{}  que si cet usage est conforme à la pratique autorisée par cet organisme.

\bigskip

\begin{rappel_enhanced}[Ressource]
Voir \textit{\color{bleudefrance}\underline{\hyperlink{http://www.naylornetwork.com/cia-nwl/articles/index-fr-v6.asp?aid=606558&issueID=63257}{\color{bleudefrance} cet article}}} concernant la bonne utilisation du titre professionnel par M. Michel Simard, D.G. de l'ICA.
\end{rappel_enhanced}
\end{definitionGENERAL}

\begin{definitionNOHFILLsub}[Annotation 10-1]
\begin{description}
	\item[Titre]	s’entend de tout titre conféré par un organisme actuariel reconnu à l’égard d’un poste particulier au sein de cet organisme. 
	\item[Désignation]	s’entend de la mention expresse du statut de membre de cet organisme. 
\end{description}
\end{definitionNOHFILLsub}

\bigskip

\begin{definitionGENERAL}{Obligations parallèles (condamnations au criminel)}[\circled{11}{black}][azure(colorwheel)]
Le membre divulgue toute condamnation au criminel conformément à l'article 3.1.12(1), 3.1.12(2) ou 3.1.12(3) des status administratifs.
\end{definitionGENERAL}

\begin{rappel}{Status administratifs 3.1.12 sur l'obligation de divulger}
\begin{description}
	\item[3.1.12(1)]	Un Fellow, un associé ou un affilié qui ne jouit pas d’une exonération de cotisation en raison d’une retraite complète conformément à l’article 7.02(b), doit divulguer au directeur général de l’Institut \textbf{toute condamnation au criminel dont il a fait l’objet} le ou après le 1er septembre 2016 et ce, dans un délai de 30 jours suivant ladite condamnation au criminel. 
	\item[3.1.12(2)]	Un Fellow, un associé ou un affilié qui ne jouit pas d’une exonération de cotisation en raison d’une retraite complète conformément à l’article 7.02(b), doit divulguer au directeur général de l’Institut, au plus tard le 1er juillet 2017, toute condamnation au criminel dont il a fait l’objet avant le 1er septembre 2016.
	\item[3.1.12(3)]	Un Fellow, un associé ou un affilié qui jouit d’une exonération de cotisation en raison d’une retraite complète conformément à l’article 7.02(b) et qui ne divulgue pas une condamnation au criminel conformément à l’article 3.1.12(1) ou 3.1.12(2), et qui par la suite cesse de jouir de l’exonération de cotisation, doit :
		\begin{enumerate}[label = (\alph*)]
		\item	divulguer au directeur général de l’Institut, dans un délai de 30 jours suivant l’avis à l’Institut de son intention de cesser son exonération de cotisation, toute condamnation au criminel dont il a fait l’objet le ou après le 1er septembre 2016; 
		\item	divulguer au directeur général de l’Institut, au plus tard le 1er juillet 2017 ou immédiatement suivant l’avis à l’Institut de son intention de cesser son exonération de cotisation, si un tel avis est reçu le ou après le 1er juillet 2017, toute
condamnation au criminel dont il a fait l’objet avant le 1er septembre 2016.
		\end{enumerate}
\end{description}
\end{rappel}

\begin{definitionNOHFILLsub}[Annotation 11-1]
\og Condamnation au criminel \fg{}  fait un renvoi à la définition de \og condamnation au criminel \fg{}  précisée dans les statuts administratifs.
\end{definitionNOHFILLsub}

\begin{definitionNOHFILLsub}[Annotation 11-2]
Une condamnation au criminel divulguée par un membre conformément à l’article 3.1.12(1), 3.1.12(2) ou 3.1.12(3) des statuts administratifs est assujettie à un examen et à une évaluation par un conseil, une direction ou une commission désignée de l’Institut qui peut déterminer que la question devrait être référée au Conseil de déontologie et soumise aux procédures disciplinaires de l’Institut.
\end{definitionNOHFILLsub}

\bigskip

\begin{definitionGENERAL}{Obligations parallèles (collaboration)}[\circled{12}{black}][azure(colorwheel)]
Le membre répond promptement, sincèrement et complètement à toute demande de renseignements reçue du Conseil de déontologie, d'une équipe d'enquête, d'un tribunal disciplinaire, d'un tribunal d'appel ou de tout membre de ces groupes, et offre son entière collaboration à ces derniers relativement à toute question disciplinaire prévue à la section 20 des status administratifs.

\bigskip

\begin{rappel_enhanced}[Exemple]
\begin{itemize}
	\item	collaborer à une enquête concernant une plainte. 
	\item	Fournir des documents même si légalement protégés.
\end{itemize}
\end{rappel_enhanced}
\end{definitionGENERAL}

\begin{definitionNOHFILLsub}[Note]
Si une plainte est déposée contre vous, il est important que vous collaboriez entièrement avec le Conseil de déontologie et à toute enquête qu’elle mènera.  

\bigskip

Il y a déjà eu des situations où les membres n’ont pas collaboré, ignorant parfois les communications et les requêtes aux fins de rencontres. Cela ne peut que nuire à votre situation et possiblement entraîner une augmentation des cotisations des membres.  

\bigskip

Même si les informations demandées sont protégées légalement et qu’un juge dans une affaire stipule que les documents ne peuvent pas être partagés, le membre est toujours tenu de travailler avec le CD pour obtenir ces documents. Nos Règles de déontologie sont une condition d’adhésion. Ne pas fournir les documents demandés ou ne pas coopérer avec le CD pourrait soumettre le membre à des procédures disciplinaires. Le CD conserve toutes les informations en toute confidentialité.
\end{definitionNOHFILLsub}

\begin{definitionGENERAL}{Obligations parallèles (non-conformité apparente)}[\circled{13}{black}][azure(colorwheel)]
Un membre qui prend connaissance d'un cas important de non-conformité apparente aux Règles ou aux normes de pratique de la part d'un autre membre doit tenter de discuter de la situation avec l'autre membre et résoudre la non-conformité apparente. En l'absence de discussion et de résolution, le membre doit signaler la non-conformité apparente au Conseil de déontologie, sauf si un tel signalement est contraire à la loi ou lorsque le membre est appelé à intervenir dans un contexte conflictuel, et ce pour la durée du contexte conflictuel. 

\bigskip

\begin{rappel_enhanced}[Exemple de non-conformité]
\begin{itemize}
	\item	conseiller un client de loger une plainte au lieu de tenter d'en discuter avec le membre
\end{itemize}
\end{rappel_enhanced}
\end{definitionGENERAL}

\end{multicols*}
%% -----------------------------
%% Fin du document
%% -----------------------------
\end{document}
