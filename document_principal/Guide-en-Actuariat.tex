\documentclass[oneside, 12pt, french]{book}
% !TEX encoding = UTF-8 Unicode
% LaTeX Preamble
% Author : Gabriel Crépeault-Cauchon
% Last update : 08/09/2019
% ---------------------------------------------
% BEGINNING OF PREAMBLE
% ---------------------------------------------
% Encoding packages
\usepackage[utf8]{inputenc}
\usepackage[T1]{fontenc}
\usepackage{babel}
\usepackage{lmodern}

% HYPERREF (URL's and Link options)
\usepackage{hyperref}
\hypersetup{colorlinks = true, urlcolor = red!60!black, linkcolor = red!60!black}

% POLICY (choose one of them)
%	\usepackage{concrete}
%	\usepackage{mathpazo}
%	\usepackage{frcursive} %% permet d'écrire en lettres attachées
 \usepackage{aeguill}
% 	\usepackage{mathptmx}
%	\usepackage{fourier} 

% Mathematics configuration
\usepackage{amsmath,amsthm,amssymb,latexsym,amsfonts}
\usepackage{empheq}
\usepackage{numprint}
\usepackage{dsfont} 


% Tcolorbox config
\usepackage{tcolorbox}
\tcbuselibrary{xparse}
\tcbuselibrary{breakable}

% Définition Boite pour exemple
\newcounter{ex}[section]
\DeclareTColorBox{exemple}{ o }% #1 parameter
{colframe=green!20!black,colback=green!5!white, % color of the box
breakable, pad at break*=0mm, % to split the box
before title = {\textbf{Exemple \stepcounter{ex} \arabic{chapter}.\arabic{section}.\arabic{ex} }},
IfValueTF = {#1}{title= {#1}}{title= \hphantom},
after title = {\large \hfill \faWrench}
}

%% Définition boite pour définition
\newcounter{def}[section]
\DeclareTColorBox{definition}{ o }% #1 parameter
{colframe=blue!60!green,colback=blue!5!white, % color of the box
breakable, pad at break*=0mm, % to split the box
before title = {\textbf{Définition \stepcounter{def} \arabic{chapter}.\arabic{section}.\arabic{def} }},
title = {#1},
after title = {\large \hfill \faBook}
}

\DeclareTColorBox{note}{ o }
    {colframe=black,
     colback=white,
     sharp corners,
     pad at break*=0mm,
     IfValueTF={#1}{title={#1}, fonttitle=\bfseries}{title=Note, fonttitle=\bfseries}}


% Graphics and picture import Packages
\usepackage{graphicx}
\usepackage{pict2e}

% insert PDF package
\usepackage{pdfpages}

% Color package
\usepackage{color, soulutf8, colortbl}

% Mathematics table
\usepackage{array}   % for \newcolumntype macro
\newcolumntype{L}{>{$}l<{$}} % math-mode version of "l" column type

% usefull shortcut for colored text
\newcommand{\orange}{\textcolor{orange}}
\newcommand{\red}{\textcolor{red}}
\newcommand{\cyan}{\textcolor{cyan}}
\newcommand{\blue}{\textcolor{blue}}
\newcommand{\green}{\textcolor{green}}
\newcommand{\darkgreen}{\textcolor{darkgreen}}
\newcommand{\purple}{\textcolor{magenta}}
\newcommand{\yellow}{\textcolor{yellow}}

% Colors define
\definecolor{darkgreen}{RGB}{37, 128, 40}
\definecolor{tocColor}{HTML}{8A2507}

% Custum enumerate & itemize Package
\usepackage{enumitem}
% French Setup for itemize function
\frenchbsetup{StandardItemLabels=true}

% Mathematics shortcut
\usepackage{cancel}
\newcommand{\reels}{\mathbb{R}}
\newcommand{\entiers}{\mathbb{Z}}
\newcommand{\naturels}{\mathbb{N}}
\newcommand{\eval}{\biggr \rvert}
\newcommand{\esp}[1]{\mathrm{E} \left[ #1 \right]} % espérance
\newcommand{\variance}[1]{\mathrm{Var} \left( #1 \right)} % variance
\newcommand{\covar}[1]{\mathrm{Cov} \left( #1 \right)} % variance
\newcommand{\prob}[1]{\Pr \left( #1 \right)} % probabilité entre parenthèses
\newcommand{\laplace}{\mathcal{L}}
\newcommand{\matr}[1]{\mathbf{#1}} % Notation matricielle
\DeclareMathOperator{\Tr}{Tr}
\newcommand{\fgp}{\mathcal{P}}
\DeclareMathOperator{\Adj}{Adj}
\newcommand{\derivee}[1]{\frac{\partial}{\partial #1}}
\newcommand{\indic}[1]{\mathds{1}_{\{ #1 \}}}
\newcommand{\VaR}[2][k]{\mathrm{VaR}_{#1}{\left( #2 \right)}}
\newcommand{\TVaR}[2][k]{\mathrm{TVaR}_{#1}{\left( #2 \right)}}


% Matricial anotation for math symbols (\bm{•})
% à enlever éventuellement, j'ai ajouté la macro \matr{} à la place.
\usepackage{bm}

% Actuarial notation package
\usepackage{actuarialsymbol}
\usepackage{actuarialangle}

% To indicate equation number on a specific line in align environment
\newcommand\numberthis{\addtocounter{equation}{1}\tag{\theequation}}

% Other shortcut
\newcommand{\p}{\paragraph{}}
\newcommand{\n}{\newline}

% source : https://tex.stackexchange.com/questions/112576/math-mode-in-tabular-without-having-to-use-everywhere



% Special symbols package
 \usepackage[tikz]{bclogo}
\usepackage{fontawesome}

% Retire l'indentation automatique de Latex
\setlength{\parindent}{0pt}

% Utilisé pour la page couverture
\usepackage[absolute]{textpos} % Textblock environement
\usepackage{anyfontsize} % Avoir un gros titre
\usepackage{titling} % Avoir un gros titre
\usepackage{changepage} % ajustwidth environement

% Pour afficher du code
\usepackage{listings}

\definecolor{codegray}{gray}{0.9}
\newcommand{\code}[1]{\colorbox{codegray}{\texttt{#1}}}

\definecolor{insideBlackTerminal}{RGB}{33,33,33}
% Set Language
% \lstset{
%     language={bash},
%     basicstyle=\small\ttfamily\color{white}, % Global Code Style
%     captionpos=b, % Position of the Caption (t for top, b for bottom)
%     extendedchars=true, % Allows 256 instead of 128 ASCII characters
%     tabsize=2, % number of spaces indented when discovering a tab 
%     columns=fixed, % make all characters equal width
%     keepspaces=true, % does not ignore spaces to fit width, convert tabs to spaces
%     showstringspaces=false, % lets spaces in strings appear as real spaces
%     breaklines=true, % wrap lines if they don't fit
%     frame=single, % draw a frame at the top, right, left and bottom of the listing
%     numberstyle=\tiny\ttfamily, % style of the line numbers
%     % commentstyle=\color{red}, % style of comments
%     % keywordstyle=\color{red}, % style of keywords
%     % stringstyle=\color{red}, % style of strings
%     backgroundcolor = \color{insideBlackTerminal},
%     rulecolor=\color{red}
% }

\usepackage{lstlinebgrd}
\definecolor{grayComment}{HTML}{8D90B8}
\lstset{
language=R,                     % the language of the code
basicstyle=\ttfamily, % the size of the fonts that are used for the code
% numbers=left,                   % where to put the line-numbers
% numberstyle=\color{blue},  % the style that is used for the line-numbers
% stepnumber=1,                   % the step between two line-numbers. If it is 1, each line
% will be numbered
numbersep=5pt,                  % how far the line-numbers are from the code
backgroundcolor=\color{white},  % choose the background color. You must add \usepackage{color}
linebackgroundcolor=\color{white},
showspaces=false,               % show spaces adding particular underscores
showstringspaces=false,         % underline spaces within strings
showtabs=false,                 % show tabs within strings adding particular underscores
frame=single,                   % adds a frame around the code
rulecolor=\color{black},        % if not set, the frame-color may be changed on line-breaks within not-black text (e.g. commens (green here))
tabsize=2,                      % sets default tabsize to 2 spaces
captionpos=b,                   % sets the caption-position to bottom
breaklines=true,                % sets automatic line breaking
breakatwhitespace=false,        % sets if automatic breaks should only happen at whitespace
%   keywordstyle=\color{functionR},      % keyword style
commentstyle=\color[HTML]{9F0808},  %\color[HTML]{8D90B8},   % comment style
%   stringstyle=\color[HTML]{1D9507},      % string literal style
moredelim=**[is][\color{grayComment}]{@}{@}, % couleur manuel
literate=%
{à}{{\`a}}1
{é}{{\'e}}1
{è}{{\`e}}1
} 




















































% ---------------------------------------------
% END OF PREAMBLE
% ---------------------------------------------

\newcommand{\TITLE}{%
Guide de survie  %The blank line after is needed.

\vspace{3mm}
en Actuariat
}
\newcommand{\AUTHORS}{%
Gabriel Crépeault-Cauchon \href{https://github.com/gabrielcrepeault}{\faGithub}  \href{https://www.linkedin.com/in/gabrielcrepeault/}{\faLinkedinSquare}  \\
Nicholas Langevin  \href{https://github.com/NicholasLangevin}{\faGithub}  \href{https://www.linkedin.com/in/nicholas-langevin-216474182/}{\faLinkedinSquare} \\
}

\begin{document}
% Page titre et table des matières
% Page couverture du document de reference

\pagestyle{empty}

\textblockorigin{0mm}{0mm}
\setlength{\parindent}{0mm}

\begin{textblock*}{\paperwidth}(0mm,0mm)
\includegraphics[height=11in, width=\paperwidth]{src/GuideEnActuariat/couverture.jpg} \\
\end{textblock*}

\begin{textblock*}{500mm}(30mm,10mm)
\sffamily
\bfseries\fontsize{52}{42}\selectfont
{ \color{white} 
\TITLE
}
\end{textblock*}

\null\newpage

% ------------------------------------------
% Deuxième page : présentation des auteurs et liens vers le dépôt Github du guide
% ------------------------------------------
\begin{textblock*}{\paperwidth}(20mm,40mm)
\raggedright
\sffamily
\bfseries\fontsize{52}{42}\selectfont
\TITLE
\end{textblock*}

\begin{textblock*}{\paperwidth}(20mm,100mm)
\bfseries\fontsize{15}{15}\selectfont
\raggedright
\AUTHORS
\end{textblock*}


\begin{textblock*}{\paperwidth}(20mm,210mm)
\raggedright
\sffamily
\bfseries\fontsize{10}{10}\selectfont
\href{https://github.com/NicholasLangevin/Guide_de_survie_en_actuariat}{Dépôt officiel de ce document} \\
Dernière mise à jour : \today
\end{textblock*}

\null\newpage


\frontmatter % PRÉFACE
\include{Main-Introduction}
\chapter{Liste des collaborateurs}
Voici la liste des personnes ayant collaboré à ce document : 
\begin{itemize}
\item Gabriel Crépeault-Cauchon
\item Nicholas Langeevin
\end{itemize}

{
    \hypersetup{linkcolor=tocColor}
    \bfseries % Pas encore sur si c'est plus beau...
    \tableofcontents
}
\newpage
% ------

\mainmatter % début du document principal

% FONDEMENTS MATHÉMATIQUES
\part{Fondements mathématiques utiles}

\include{Chapter-NotionCalculs}
\label{Chapter:NotionCalculs}

\include{Chapter-AlgebreLineaire}
\label{Chapter:AlgebreLineaire}


% UNIVERSITÉ
\part{Matière vue dans le baccalauréat en actuariat}

% Chapitre sur les concepts de probabilité (ACT-1002) et statistiques (ACT-2000)
\chapter{Probabilités et statistiques}

\section{Concepts de probabilité de base}

\subsection{Probabilité conditionnelle}
\paragraph{Définition de base}
\begin{equation}
\label{eq:prob-cond}
\prob{A|B} = \frac{\prob{A \cap B}}{\prob{B}}
\end{equation}

\paragraph{Loi des probabilités totales} Soit $E_i$ le \textit{outcome} $i$ parmi l'ensemble des $n$ \textit{outcome} possibles de l'évènement $E$, alors, on peut représenter la probabilité que l'évènement $A$ survienne comme
\begin{equation}
\label{eq:loi-prob-totales}
\prob{A} = \sum_{i=1}^{n} \prob{A | E_i} \prob{E_i}
\end{equation}
avec $\sum_{i=1}^{n} \prob{E_i} = 1$.

\paragraph{Relation importante} de l'\autoref{eq:prob-cond}, on peut représenter $\prob{A|B}$ comme
\begin{equation}
\label{eq:prob-cond-2}
\prob{A|B} = \frac{\prob{B|A} \prob{A}}{\prob{B}}
\end{equation}

\subsection{Théorème de Bayes} En combinant l'\autoref{eq:prob-cond-2} et la loi des probabilités totales (l'\autoref{eq:loi-prob-totales}), on obtient le théorème de Bayes : 
\begin{equation}
\prob{A|B} = \frac{\prob{B|A} \prob{A}}{\sum_{i=1}^{n} \prob{B | A_i} \prob{A_i}}
\end{equation}


\section{Définition d'une variable aléatoire}

\section{Distribution d'une variable aléatoire}
Fonction de densité, répartition, survie, hazard rate, etc.

\section{Moments et quantités importantes}
Espérance, variance, covariance, coefficient de variation, corrélation

\paragraph{Espérance} Soit une v.a. $X$ (continue ou discrète). Son espérance est définie telle que
\begin{equation}
\label{eq:esp-univarie}
\esp{X} = \mu = \sum_{x=0}^{\infty} x \prob{X = x} = \int_{0}^{\infty} x f_X(x) dx
\end{equation}
L'espérance d'une fonction de la v.a $X$ est
\begin{equation}
\label{eq:esp-fct-univarie}
\esp{g(X)} = \sum_{x=0}^{\infty} g(x) \prob{X = x} = \int_{0}^{\infty} g(x) f_X(x) dx
\end{equation}

\paragraph{Variance}
\begin{equation}
\label{eq:variance}
\variance{X} = \sigma^2 = \esp{(X - \esp{X})^2} = \esp{X^2} - \esp{X}^2
\end{equation}
quelques propriétés à savoir : 
\begin{align*}
\variance{aX} 		& = a^2 \variance{X} \\
\variance{X + b}	& = \variance{X}
\end{align*}

\paragraph{Covariance}
\begin{align}
\label{eq:covariance}
\covar{X,Y} &=  \sigma_{X,Y} \\
            &= \esp{(X-\esp{X})(Y - \esp{Y})} \\
            &= \esp{XY} - \esp{X} \esp{Y}
\end{align}



\section{Distribution de probabilité qui reviennent souvent}
Un tableau récapitulatif des différentes distribution de probabilité est disponible à l'

\label{Chapter:ProbEtStatistique}

\chapter{Mathématiques financières}
to-do
\label{Chapter:MathematiqueFinanciere}

\include{Chapter-ProcessusAleatoires}
\label{Chapter:ProcessusAleatoires}

\chapter{Théorie du risque}


\section{Modèle pour les risques et méthodes d'estimation}
\paragraph{Introduction} 
Dans le cours \emph{Introduction à l'actuariat II}, on a vu comment produire des réalisation $x^{(j)}$ de $x$ du modèle fréquence-sévérité, dans le cas ou $B\sim$ Gamma. Dans ce cours, on développe des technique récursive pour évaluer la convolution.

\subsection{Méthode d'estimation}

\paragraph{Context \#1}
\begin{enumerate}[label=(\arabic*)]
    \item Pour chaques contrats $(j)$, on dispose du nombre de sinistres $(n_j)$ et des montants de chaques sinistres $(y_1,...,y_{n_j})$.
    \item On définie \[ X_j = \sum_{k=1}^{n_j} y_{j,k} \] où $X_j = 0$ si $n_j = 0$.
    \item On pose $\underline{\theta}^N$ et $\underline{\theta}^Y$, les paramètres à estimer. On utilise la méthode du maximum de vraisemblance pour estimer ses paramètres. 
        \begin{align*}
            \mathcal{L}(\underline{\theta}^N,\underline{\theta}^Y) &= \prod_{j=1}^m \left\{ f_n(n_j|\underline{\theta}^n) \prod_{k=1}^{n_j} f_y(y_{j,k}|\underline{\theta}^y) \right\} \\
            &= \left( \prod_{j=1}^m f_n(n_j|\underline{\theta}^n) \right) \left( \prod_{j=1}^m \prod_{k=1}^{n_j} f_y(y_{j,k}|\underline{\theta}^y) \indic{n_j>0} \right) \\
            &= 
            &= \mathcal{L}(\underline{\theta}^N)\mathcal{L}(\underline{\theta}^Y)
        \end{align*}
    \item Remarque:
    \begin{itemize}
        \item Le résultat découle de l'indépendence entre $N$ et $\underline{Y}$.
        \item Ce résultat facilite l'estimation.
        \item On peut estimer des paramètres avec des lois de fréquence et sévérité séparément.
    \end{itemize}
\end{enumerate}

\paragraph{Context \#2}
\begin{enumerate}[label=(\arabic*)]
    \item Pour chaque contrat $(j)$, on dispose du nombre de sinistres $(n_j)$ et si le nombre de sinistre est non null, on connait le montant \textbf{total} des sinistres. On ne connait pas les montants de chaques sinistre.
    \item On pose $\underline{\theta}^N$ et $\underline{\theta}^Y$, les paramètres à estimer. On utilise la méthode du maximum de vraisemblance pour estimer ses paramètres. 
        \begin{align*}
            \mathcal{L}(\underline{\theta}^N,\underline{\theta}^Y) &= \prod_{j=1}^m  f_n(n_j|\underline{\theta}^n)  f_{Y_1+...+Y_{n_j}}(x_j|\underline{\theta}^y) \indic{n_j>0} \\
            &= \mathcal{L}(\underline{\theta}^N)\mathcal{L}(\underline{\theta}^{Y_1+...+Y_{n_j}})
        \end{align*}
    \item Remarque:
    \begin{itemize}
        \item Le résultat découle de l'indépendence entre $N$ et $\underline{Y}$.
        \item Ce résultat facilite l'estimation.
        \item On peut estimer des paramètres avec des lois de fréquence et sévérité séparément si on connait la loi de $Y_1+...+Y_{n_j}$.
    \end{itemize}
\end{enumerate}

\paragraph{Context \#3}
\begin{enumerate}[label=(\arabic*)]
    \item Pour chaque contrat $(j)$, on connait uniquement les coûts totaux, null ou non null.
    \item On pose $\underline{\theta}^N$ et $\underline{\theta}^Y$, les paramètres à estimer. On utilise la méthode du maximum de vraisemblance pour estimer ses paramètres. 
    \item Possibilités \#1, modèle forfaitaire, $X_j = C \cdot \indic{I=1}$.
        \begin{align*}
            \mathcal{L}(\underline{\theta}^N,\underline{\theta}^Y) &= \prod_{j=1,x_j=0}^m f_I(0|\underline{\theta}^I) \prod_{j=1,x_j>0}^m f_I(1|\underline{\theta}^I) \cdot f_C(x_j|\underline{\theta}^C) 
        \end{align*}
    \item Possibilités \#2, modèle fréquence-sévérité. Si on connait la loi de la somme $(Y_1+...+Y_k)$. La distribution est donc mixte avec masse de probabilité à 0. Ainsi on a
    \begin{align*}
        f_X(0|\underline{\theta}^N,\underline{\theta}^Y) &= \prob{N=0|\underline{\theta}^N} \\
        f_X(x_j|\underline{\theta}^N,\underline{\theta}^Y) &= \sum_{k=1}^\infty \prob{N=k|\underline{\theta}^N} f_{Y_1+...+Y_k}(x_j|\underline{\theta}^Y) \\
        \mathcal{L}(\underline{\theta}^N,\underline{\theta}^Y) &= \prod_{j=1,x_j=0}^m \prob{N=0|\underline{\theta}^N} \prod_{j=1,x_j>0}^m f_X(x_j|\underline{\theta}^N,\underline{\theta}^Y)
    \end{align*}
    \item Remarque: Contrairement au deux autre context, on ne peut pas estimé séparément $\underline{\theta}^N$ et $\underline{\theta}^Y$.
\end{enumerate}

\section{Processus Stochastique}

\subsection{Processus de poisson homogène}
\begin{figure}
    \centering
    \includegraphics{src/TheorieDuRisque/Def-ProcessusPoissonHomogene.PNG}
\end{figure}

\subsection{Processus Homogène Composée}

\paragraph{Definition}
\[ S(t) = \sum_{k=1}^{N(t)} X_k \]

\paragraph{Fonction de répartition}
\[ F_{S(t)}(x) = \prob{N(t) = 0} + \sum_{k=1}^\infty \prob{N(t) = k} * F_{X_1+...+X_k}(x) \]
\begin{lstlisting}[language=R, caption={Exemple Pois-Gamma}]
F_s <- function(x, t){
    dpois(0, lambda * t) + sum(sapply(1:k0, function(k) dpois(k, lambda * t) * pgamma(x, alpha * k, beta)
}
\end{lstlisting}

\paragraph{Value at risk}
\[ \VaR{S(t)} = F_{S(t)}^{-1}(k) \]

\begin{lstlisting}[language=R, caption={Exemple Pois-Gamma}]
VaR_s <- function(kappa, t){
    if(kappa <= dpois(0, lambda * t)
        return(0)
    uniroot(function(x) F_s(x, t) - kappa, c(0, 10000))$root
}
\end{lstlisting}

\paragraph{Tail Values at Risk}
\[ \TVaR{S(t)} = \sum_{k=0}^\infty \prob{N(t) = k} \cdot \TVaR{X_1+...+X_k}\]

\begin{lstlisting}[language=R, caption={Exemple Pois-Gamma}]
TvaR_S <- function(kappa, t){
    sum(sapply(1:k0, function(k) dpois(k, 1.8 * t) * alp    ha * k / beta * (1 - pgamma(VaR_s(kappa, t), (alpha*k)+1, beta)))) / (1 - kappa)
}
\end{lstlisting}

\subsection{Processus Poisson Mixte}

\paragraph{Definition}
Soit $\Lambda$ une variable aléatoire positive (continue ou discrète). Si le
processus de comptage $\underline{N} = \{N(t);t \geq 0\}$ étant donné que $\Lambda = \lambda$ est
un processus de Poisson de taux $\Lambda$ alors $\underline{N} = \{N(t);t \geq 0\}$ est appelé
un processus de Poisson mixte.  \\ 

Les accroissements du processus de Poisson mixte $\underline{N}$ sont \textbf{indépendant} et \textbf{stationnaire}. \\
\paragraph{Preuve (stationnaire)}
    \begin{align*}
        M_{N(t, t+s]}(r) &= \esp{e^{r N(t, t+s}} \\
                         &= \esp[\Lambda]{\esp{e^{r N(t, t+s]}|\Lambda}} \\
                         &= \esp[\Lambda]{e^{\Lambda t(e^r - 1)}} \\
                         &= M_\Lambda(t(e^r - 1)) \\
                         &= M_{N(t)}(r) \\
                         &= \text{Stationnaire car fonction de $t$ seulement}
    \end{align*}

\paragraph{Preuve (indépendance)}
A faire

\begin{note}
    Les temps-inter siniste sont échangeable, mais ne sont pas indépendant. Par contre,
    les temps-inter siniste $(W_1|\Lambda)$ et $(W_2|\Lambda)$ sont conditionnelement indépendant et $(w|\Lambda) \sim \text{Exp}(\lambda)$. \\
\end{note}

\begin{figure}[!ht]
    \centering
    \includegraphics[scale=0.5]{src/TheorieDuRisque/SimulationProcessusPoisson.png}
    \includegraphics[scale=0.5]{src/TheorieDuRisque/SimulationProcessusPoissonMixte.png}
    \caption{Comparaison entre un processus de Poisson homogène et un processus de poisson mixte. Le graphique du bas représente le processus mixte.}
\end{figure}

\begin{align*}
    \prob{N(t) = n} &= \int_{-\infty}^\infty \prob{N(t) = n | \Lambda} \cdot f_\Lambda(\lambda)\: d\lambda \\ 
    &\text{si } \Lambda \sim \Gamma(\alpha, \beta) \\
    &= \frac{\Gamma(\alpha+n)}{\Gamma(\alpha)k!} \left( \frac{\beta}{\beta + t} \right)^\alpha \left( \frac{1}{\beta + t} \right)^n \sim \text{BinNeg}(\alpha, \frac{\beta}{\beta + t})
\\
    \prob{N(t, t+s] = n} &= \int_{-\infty}^\infty \prob{N(t, t+s] = n | \Lambda} \cdot f_\Lambda(\lambda)\: d\lambda \\
    &= \int_{-\infty}^\infty \prob{N(s) = n | \Lambda} \cdot f_\Lambda(\lambda)\: d\lambda 
\\ \\
    \prob{N(t) = k_1, N(t, t+s] = k_2} &= \int_{-\infty}^\infty \prob{N(t) = k_1, N(t, t+s] = k_2| \Lambda} \cdot f_\Lambda(\lambda)\: d\lambda \\
    &= \int_{-\infty}^\infty \prob{N(t) = k_1 | \Lambda} \prob{N(s) = k_2 | \Lambda}\cdot f_\Lambda(\lambda)\: d\lambda
\end{align*}
\begin{align*}
    \esp{N(t+s)|N(t) = k_1} &= \esp{N(t) + N(t, t+s] | N(t) = k_1} \\
    &= k_1 + \esp[\Lambda]{\esp{N(t, t+s]|N(t) = k_1, \Lambda}|N(t) = k_1} \\
    &= k_1 + \esp[\Lambda]{\lambda t|N(t) = k_1} \\
    &= k_1 + \int_{-\infty}^\infty \lambda t \frac{f_{\Lambda, N(t)}(\lambda, k_1)}{\prob{N(t) = k_1}}\: d\lambda \\
    &= k_1 + \frac{\int_{-\infty}^\infty\lambda t \prob{N(t) = k_1|\Lambda} f_\Lambda(\lambda)\:d\lambda}{\int_{-\infty}^\infty \prob{N(t) = k_1|\Lambda} f_\Lambda(\lambda)\:d\lambda} \\
    &\text{si } \Lambda \sim \Gamma(\alpha, \beta) \\
    &= k_1 + \frac{\alpha + n}{\beta + t}
\end{align*}
\begin{align*}
    F_{W_1}(t) &= \int_{-\infty}^\infty F_{w_1|\Lambda}(t) f_\Lambda(\lambda)\,d\lambda \\
    &\text{si } \Lambda \sim \Gamma(\alpha, \beta) \\
    &= \left( \frac{\beta}{\beta + t} \right)^\alpha \sim \text{Pareto}(\alpha, \beta)
\\ \\
    F_{W_1, W_2}(t_1, t_2) &= \int_{-\infty}^\infty F_{W_1}(t_1) F_{W_2}(t_2) f_\Lambda(\lambda)\,d\lambda\\
    &\text{si } \Lambda \sim \Gamma(\alpha, \beta) \\
    &= \left( \frac{\beta}{\beta + t_1 + t_2 } \right)^\alpha \sim \text{Pareto Mulivarié}
\end{align*}

\label{Chapter:TheorieDuRisque}


% EXAMENS PROFESSIONNELS
\part{Matière pour les examens professionnels}

{ \color{red} This section resume the Chapters 1-5 (excluding Sections 3.3 and 3.4), 6, 7 (Sections 7.1, 7.2 and 7.3) of \emph{An Introductory Time Series with R} }

\hl{A mettre dans la bio: Cowpertwait, P. and Metcalfe, A., Introductory Time Series with R, Springer, 2009.}

\paragraph{Trend}
In general, a systematic change in a time series that does not appear to be periodic is known as a trend.

\paragraph{Seasonal Variation}
A repeating pattern within any fixed period.

\paragraph{Notation}
We represent a time series of length n by ${x_t : t = 1,...,n} = {x_1, x_2, ..., x_n}$

\paragraph{Forecast}
A forecast $\hat{x}_{t+k|t}$ is a predicted future value, and the number of time steps into the future is the \textbf{lead time (k)}

\section{Base Models}

\paragraph{Additive Decomposition}
The additive decomposition model is given by
\[ x_t = m_t + s_t + z_t \]
where $t$ is the time, $x_t$ the observed series, $m_t$ the trend, $s_t$ the seasonal effect and $z_t$ the error term.

\paragraph{Multiplication Model}
If the seasonal effect tends to increase as the trend inscrease, we use a multiplication model define as
\[ x_t = m_t \cdot s_t + z_t \]
where $t$ is the time, $x_t$ the observed series, $m_t$ the trend, $s_t$ the seasonal effect and $z_t$ the error term.

\subsection{Estimating Trends}

\paragraph{Centered Moving Average}
A moving average is an average of a specified number of time series values around each value in the time series, with the exception of the first few and last few terms.
The length of the moving average is chosen to average out the seasonal effects, which can be estimated
\[ \hat{m}_t = \frac{0.5m_{t-k} + m_{t-k-1} + ... + m_t + ... + m_{t+k-1} + 0.5m_{t+k}}{2k} \]

\subsection{Estimating Seasonal Effect}

\paragraph{Additive Effect}
An estimate of the monthly additive effect ($s_t$) at time t is define by
\[ \hat{s}_t = x_t - \hat{m}_t \]
It is usual to adjust those estimate in order that the sum of one period of the time serie equal zero. Let $c$ be that adjustment in order to solve this expression.
\[ \sum (s_t + c) = 0 \]

\paragraph{Multiplicative Effect}
An estimate of the monthly multiplicative effect ($s_t$) at time t is define by
\[ \hat{s}_t = \frac{x_t}{\hat{m}_t} \]
And we found the ajustment $c$ in order to solve this expression.
\[ \sum \frac{(\hat{s}_t + c)}{n} = 1 \] 

\begin{figure}[!ht]
    \centering 
    \includegraphics[scale=0.6]{src/SerieChronologique/DecompositionMultiTimeSerie.png}
    \caption{In this example, the multiplicative model would seem more appropriate than the additive model because the variance of the original series and trend increase with time} 
\end{figure}

\subsection{Smoothing Procedure}
Smoothing procedures can, and usually do, use points before and after the time at which the smoothed estimate is to be calculated. A consequence is that the smoothed series will have some points missing at the beginning and the end unless the smoothing algorithm is adapted for the end points.
\begin{itemize}
    \item[Ex.1] The centering moving average is an exemple of a \emph{smoothing} procedure. 
    \item[Ex.2] The \emph{loess} technique is also a \emph{smoothing} that use a locally weighted regression.

\end{itemize}

\section{Correlation}

\paragraph{Mean Function}
The mean function is, in general, a function of t and it define as 
\[\mu(t) = \esp{x_t} \]

\paragraph{Sample Mean}
The sample mains is define as
\[ \bar{x} = \sum \frac{x_i}{n} \]

\paragraph{Variance Function}
The variance function of a time series model that is stationary in the mean is
\[ \sigma^2(t) = \esp{(x_t - \mu^2)} \]

\paragraph{Sample Variance}
If the model is stationary in the variance, we can estimate the variance with the sample variance define as
\[ \variance{x} = \frac{\sum (x_t - \bar{x}}{n-1} \]

\paragraph{Stationarity}
If the mean function is constant, we say that the time series model is stationary in the mean. The time serie can also be stationary in the variance, if the variance function is constant $\sigma^2$.

\paragraph{Ergodic Serie}
A time series model that is stationary in the mean is ergodic in the mean if the time average for a single time series tends to the ensemble mean as the length of the time series increases.
\[ \lim_{n \to \infty} \bar{x} = \mu \]

\paragraph{Covariance}
The covariance measure the \emph{linear association} between two random variables. The covariance is define as
\[ \gamma(x,y) = \esp{(x - \mu_x)(y - \mu_y)} \]

\paragraph{Sample Covariance}
We can estimate the covariance with the sample covariance define as
\[ \covar{x, y} = \sum \frac{(x_i - \bar{x})(y_i - \bar{y})}{n-1} \]

\paragraph{Correlation}
Correlation is a dimensionless measure of the linear association between a pair of variables (x,y) and is obtained by standardising the covariance.
\[ \rho(x, y) = \frac{\esp{(x-\mu_x)(y-\mu_y)}}{\sigma_x \sigma_y} = \frac{\gamma(x,y)}{\sigma_x \sigma_y} \]
 
\paragraph{Sample Correlation}
We can estimate the covariance with the sample covariance define as
\[ \mathrm{Cor}(x, y) = \frac{\covar{x, y}}{\mathrm{sd}(x)\mathrm{sd}(y)} \]

\paragraph{Autocovariance (acvf)}
If a time series model is second-order stationary, we can define an autocovariance function (acvf), $\gamma_k$ , as a function of the lag $k$.
\[ \gamma_k = \esp{(x_t-\mu)(x_{t+k}-\mu)} \]

\paragraph{Sample Autocorrelation}
The acvf function can be estimated by
\[ c_k = \frac{1}{n} \sum_{t=1}^{n-k} (x_t - \bar{x})(x_{t+k} - \bar{x}) \]

\begin{note}
The sample autocovariance at lag 0, $c_0$, is the variance calculated with a denominator $n$. 
\end{note}

\paragraph{Autocorrelation (acf)}
The lag $k$ autocorrelaton function (acf), $\rho_k$, is defined by
\[ \rho_k = \frac{\gamma_k}{\sigma^2} \]
where $\rho_0 = 1$

\paragraph{Sample Autocorrelation}
The acf function can be estimated by
\[ r_k = \frac{c_k}{c_0} \]

\paragraph{Second-order Stationary}
Suppose a the time serie that is stationary in the mean and the variance. Then, the time serie model is second-order stationary if the correlation between variable depends only on the number of time steps separating them.


\subsection{The Correlogram}
The correlogram is plot of $r_k$ against k. The main use of the correlogram is to detect autocorrelations in the time series after we have removed an estimate of the trend and seasonal variation.

\begin{itemize}
    \item[\textbullet] The x-axis gives the lag ($k$) and the y-axis gives the autocorrelation ($\rho_k$) at each lag. The unit of lag is the sampling interval. Correlation is dimensionless, so there is no unit for the y-axis.
    \item[\textbullet] The dotted lines on the correlogram are drawn at 
        \[ -\frac{1}{n} \pm \frac{2}{\sqrt{n}} \]
    \item[] If the sample acf($r_k$) falls outside these line, we have evidence against the null hypothesis that $p_k = 0$ at the 5\% level.
    \item[\textbullet] The lag at 0 is always.
\end{itemize}

\begin{figure}[!ht]
    \centering 
    \includegraphics[scale=0.7]{src/SerieChronologique/Correlogram.png}
    \caption{Correlogram}
\end{figure}


\subsection{Covariance of sums of random variables}
Let $x_1, x_2,...,x_n$ and $y_1, y_2,..., y_m$ be random variables. Then 
\[ \covar{\sum_{i=1}^{n} x_i, \sum_{j=1}^{m}} y_j = \sum_{i=1}^{n} \sum_{j=1}^{m} \covar{x_i, y_i} \]
The result tells us that the covariance of two sums of variables is the sum of all possible covariance pairs of the variables.

\section{Forecasting Strategies}

\subsection{Relationships of different time serie}

\paragraph{Cross-covariance (ccvf)}
The cross-correlation (ccvf) between two time series is define as
\[ \gamma_k(x, y) = \esp{(x_{t+k} - \mu_x)(y_t - \mu_y)} \]

\begin{note}
    This is not a symmetric relationship, and the variable $x$ is lagging variable $y$ by $k$.
    \[\gamma_k(x, y) = \gamma_{-k}(y, x) \]
\end{note}

\paragraph{Sample Cross-covariance}
The ccvf function can be estimed by
\[ c_k(x, y) = \frac{1}{n} \sum_{t=1}^{n-k} (x_{t+k} - \bar{x})(y_k - \bar{y}) \]

\paragraph{Cross-correlation (ccf)}
The cross-correlation (ccf) between two time serie is define as 
\[ /rho_k(x, y) = \frac{\gamma_k(x, y)}{\sigma_x\sigma_y} \]

\begin{note}
    This is not a symmetric relationship, and the variable $x$ is lagging variable $y$ by $k$.
    \[\rho(x, y) = \rho_{-k}(y, x) \]
\end{note}

\paragraph{Sample Cross-correlation}
The ccf function can be estimed by
\[ r_k(x, y) = \frac{c_k(x, y)}{\sqrt{c_0(x, x) x_0(y, y)}} \]

\section{Basic Stochastic Models}

We may consider a \textbf{trend} to be \textbf{stochastic} when it shows inexplicable changes in direction. Those type of trend can be simulated by the models of this section.

\subsection{White Noise}

\paragraph{Residual Error}
A residual error is the difference between the observed value and the model predicted value at time t. Then the residual error, $x_t$, is defined by
\[ x_t = y_t - \hat{y}_t \]
where $y_t$ is the observed value and $\hat{y_t}$ the predicted value.

As the residual errors occur in time, they form a time series: $x_1, x_2,..., x_n$.

\paragraph{Definition} 
A white noise is a time serie $\{w_t : t = 1,2,...,n\}$ where each term is independent and with a constant variance $\sigma^2$.
\[ w_t \sim \mathrm{N}(0, \sigma^2) \]

\begin{figure}[!ht]
    \centering
    \includegraphics[scale=0.7]{src/SerieChronologique/WhiteNoise-Correlogram.png}
    \caption{Correlogram of a simulated white noise series. The underlying autocorre- lations are all zero (except at lag 0); the statistically significant value at lag 7 is due to sampling variation.}
\end{figure}

\subsection{Random Walks}

\paragraph{Definition}
Let $\{x_t\}$ be a time series. Then $\{x_t\}$ is a random walk if 
\[ x_t = x_{t-1} + w_t \]
where $\{w_t\}$ is a white noise.

\paragraph{Backward Operator}
The backward operator(or lag operator)  $\mathbf{B}$ is defined by
\[ \mathbf{B}^n = x_{t-n} \]

\paragraph{Second-order Properties}
The covariance is in function of time, so a random walk is non-stationary. This model is only suitable for short term predictions.
\begin{align*}
        \hspace*{1cm}
      \mu(t) &= 0 \\
      \sigma^2(t) &= t\sigma_w^2 \\
      \gamma_k(t) &= t\sigma_w^2 \\
      \rho_k(t) &= \frac{1}{\sqrt{1 + \frac{k}{t}}}
\end{align*} 

\paragraph{The Difference Operator}
The defference operator is defined by
\[ \nabla x_t = x_t - x_{t-1} \] 
We can also expresse it with the Backward operator
\[ \nabla^n x_t  = (1 - \mathbf{B})^n x_t \]
\begin{note}
Differencing adjacent terms of a series can transform a non-stationary series to a stationary series.
    \[ x_t - x_{t-1} = w_t \]
Here differencing two non-stationary random walk give a stationary white noise.
\end{note}

\begin{figure}[!ht]
    \centering
    \includegraphics[scale=0.7]{src/SerieChronologique/RandomWalk-Correlogram.png}
    \caption{The correlogram for the simulated random walk. A gradual decay from a high serial correlation is a notable feature of a random walk series.}
\end{figure}

\paragraph{Random Walk with Drift}
A Random walk with drif is a random walk with a mean. it defined as
\[ x_t = \delta + x_{x-1} + w_t \] 

\subsection{Autoregressive Models}

\paragraph{Definition}
The series $\{xt\}$ is an autoregressive process of order p, abbreviated to $\mathrm{AR(p)}$, if
\[ x_t = \alpha_1 x_{t-1} + \alpha_2 x_{x-2} + ... + \alpha_p x_{t-p} + w_t \] 
where $\alpha_i$ are the model parameters with $\alpha_p \neq 0$.

The following points should be noted:
\begin{itemize} 
    \item[\textbullet] The random walk is a special case of $\mathrm{AR(1)}$ with $\alpha_1 = 1$.
    \item[\textbullet] The model is a regression of $x_t$ on past terms from the same series, that where the name \emph{autoregressive} come.
    \item[\textbullet] A prediction at time $t$ is given by
        \[ \hat{x}_t = x_t = \alpha_1 x_{t-1} + \alpha_2 x_{x-2} + ... + \alpha_p x_{t-p} \]
    \item[\textbullet] The model parameters can be estimated by minimising the sum of squared errores.
\end{itemize}

\paragraph{Stationary and non-stationary AR processes}
We can expressed a AR model as a polynomial of order p in terms of the backward shift operator
\[ \theta_p(\mathrm{B})x_t = (1 - \alpha_1\mathrm{B} - \alpha_2 \mathrm{B}^2 - ... - \alpha_p \mathrm{B}^p)x_t = w_t \]
Then a AR model is stationary if all root of $\theta_p(\mathrm{B})$ exceed one in absolute values. In other word if 
\begin{center}
if all $|\mathrm{B_i}| > 1$ then the model is stationary \\
otherwise, the model is non-stationary
\end{center}

\paragraph{Second-order properties of an AR(1) model}
    \begin{align*}
      \mu_k &= 0 \\  
      \gamma_k &= \frac{\alpha^k \sigma_w^2}{1 - \alpha^2} \\
      \rho &= \alpha^k
\end{align*}

\begin{figure}[!ht]
    \centering
    \includegraphics[scale=0.7]{src/SerieChronologique/AR-Correlogram.png}
    \caption{For an AR model, the correlograms decays to zero. The correlogram decays to zero more rapidly for small $\alpha$.}
\end{figure}

\paragraph{Partial Autocorrelation}
An AR(p) process has a correlogram of partial autocorrelation $\alpha_k$ that is zero after lag p.

\section{Regression}

When we have some plausible physical explanation for a \textbf{trend} we will usually wish to model it in some \textbf{deterministic} manner. Therefore, the model of this section can be used.
\\
\\
The difference with \textbf{deterministics trend} is the when we make short term forecast, we assume that the trend will change slowly.
\\
\\
Time series regression usually differs from a standart regression analysis because the residual form a time serie and therefore tend to be serially correlated. When this correlation is possitive, the estimated standard errors of the parameter estimates, read from the computer output of a standard re- gression analysis, will tend to be less than their true value.

\subsection{Linear Models}

\paragraph{Definition}
A model for a time series $\{ x_t : t = 1,...,n\}$ is linear if it can be expressed as
\[ x_t = \alpha_0 + \alpha_1 u_{1,t} + \alpha_2 u_{2,t} + ... + \alpha_m u_{m,t} + z_t \]
where $u_{i,t}$ is the value of the i\up{th} explanatory variable, $\alpha_i$ the estimated model parameters, $z_t$ the error at time $t$.

An example of a linear model is the pth-order polynomial function of t:
\[ x_t = \alpha_0 + \alpha_1 t + \alpha_2 t^2 + ... + \alpha_p t^p + z_t \]

\begin{note} 
    Note that the errors form a time series $\{z_t\}$, with mean 0, that does not have to be Gaussian or white noise.
\end{note} 

\paragraph{Stationarity}
Linear models for time series are non-stationary when they include functions of time.

Differencing can remove both stochastic and deterministic trends from time series. Then for a polynomial of roder $m$, the mth-order differencing is required to remove the trend.
\paragraph{Autocorrelation variance estimation of the sample mean}
Let $\{x_t : t = 1,..., n\}$ be a stationary time serie with mean $\mu$, variance $\sigma^2$ and autocovariance $\covar{x_t, x_{t+k}}$. Then the variance of the sample mean is given by
\[ \variance{\bar{x}} = \frac{\sigma^2}{n} \left[ 1 + 2 \sum_{k=1}^{n-1} (1 - \frac{k}{n}) \rho_k \right] \] 

\paragraph{Generalised least squares}
A fitting procedure known as generalised least squares (GLS) can be used to provide better estimates of the standard errors of the regression parameters to account for the autocorrelation in the residual series.

\subsection{Linear Models with Seasonal Variables}

\paragraph{Additive Seasonal Indicator Variables}
To include seasonal effect, we change the constant $\alpha_o$ depending on the season. Let $s$ be the time serie measured (Ex. for monthly time serie, $s=12$). Then for each $s$, we fit a constant term.
\[ x_t = s_t + m_t + z_t \]
where $s_t$ is the seasonal constant when t falls in the i\up{th} season, $m_t$ a linear model for the trend and $z_t$ the error term.

\paragraph{Harmonic Seasonal models}
The advantage of this model is that we can represente the seasonal effect with something that is smoothly. For a time series $\{x_t\}$ with $s$ season, there are $\lfloor s/2 \rfloor$ possible cycles. The harmonic model is defined by 
\[ x_t = m_t + \sum_{i=1}^{\lfloor s/2 \rfloor} \left\{ s_i \sin(\frac{2\pi it}{s}) + c_i \cos(\frac{2\pi it}{s}) \right\} + z_t \]
where $m_t$ is a trend model \textbf{that include a constant term} ($\alpha_0$) and $s_i$ and $c_i$ are unknown parameters. 

\begin{figure}[!ht]
    \centering 
    \includegraphics[scale=0.7]{src/SerieChronologique/HarmonicModel.png}
    \caption{Two possible underlying seasonal patterns for monthly series based on the harmonic model (Equation (5.10)). Plot (a) is of the first harmonic over a year and is usually too regular for most practical applications. Plot (b) is of the same wave but with a further two harmonics added. Plot (b) illustrates just one of many ways that an underlying sine wave can be perturbed to produce a less regular, but still dominant, seasonal pattern of period 12 months.}
\end{figure}

\subsection{Forecasting from regression}
When we predic a regression time series, we try to predict in the future. The problem is that the trend might change. Therefore, it is better ot think of a forecast from a regression model as an expected value conditional on past trends continuing into the future.

\paragraph{Bias Correction} 
The process of transforming the model introduce some bias in the mean. We need to apply a correction to the mean. Note that this correction doesn't need to be apply in simulation.
\[ \hat{x}_t' = \hat{x}_t * \text{Correction} \]

\subparagraph{Lognormal Correction}
\[ e^{\sigma^2 /2} \]
\subparagraph{Empirical Correction}
\[ \frac{1}{n} \sum e^{z_t} \]

\section{Stationary Models}
Sometime, the residual will be correlated in time, as this is not accounted in the fitted regression model, we need other model. 

\subsection{Strictly Stationary Series}
A time series model $\{ x_t \}$ is \emph{strictly stationry} if the joint statistical distribution $x_{t_1},...,x_{t_n}$ is the same as the joint distribution of $x_{t_1 + m},...,x_{t_n + m}$ for all $t_1, ..., t_n$ and $m$, so that the distribution is unchanged after an arbitrary time shift.

\begin{note}
    Note that strict stationarity implies that the mean and variance are constant in time and that the autocovariance $ \covar{x_t, x_s} = \gamma_k$ (i.e. only depend on the lag $k$). If a series is not strictly stationary but the mean and variance are constant in time and the autocovariance only depends on the lag, then the series is called \emph{second-order stationary}.
\end{note}
We focus on the second-order properties in this chapter, but the stochastic processes discussed are strictly stationary.
\\ 
\\
Stationarity is an idealisation that is a property of models. If we fit a stationary model to data, we assume our data are a realisation of a stationary process. So our first step in an analysis should be to check whether there is any evidence of a trend or seasonal effects and, if there is, remove them. Regression can break down a non-stationary series to a trend, seasonal components, and residual series. It is often reasonable to treat the time series of residuals as a realisation of a stationary error series. Therefore, the models in this chapter are often fitted to residual series arising from regression analyses.

\subsection{Moving average models}

\paragraph{Definition}
 moving average (MA) process of order q is a linear combination of the current white noise term and the q most recent past white noise terms and is defined by
\[ x_t = w_t + \beta_1 w_{t-1} + ... + \beta_q w_{t-q} = \phi_q(\mathrm{B}) w_t \]
where $\phi_q$ is a polynomial of order q. Because MA processes consist of a finite sum of stationary white noise terms, they are stationary and hence have a time-invariant mean and autocovariance.

\paragraph{Second-order properties}
\begin{align*}
    \mu &= 0 \\
    \sigma^2 &= \sigma_w^2 (1 + \beta_1^2 + ... + \beta_q^2) \\
    \gamma_k &= \sigma_w^2 \sum_{i=0}^{q-k} \beta_i \beta_{i+k} \\
    \rho_k &= \frac{\sum_{i=0}^{q-k} \beta_i \beta_{i+k}}{\sum_{i=0}{q} \beta_i^2 }
\end{align*}

\paragraph{Invertible properties}
An MA process is invertible if it can be expressed as a stationary AR($\infty$) process of infinite order without an error term.
\[ w_t = (1 - \beta \mathrm{B})^{-1} x_t = x_t + \beta x_{t-1} + \beta^2 x_{t-2} + ... \]
provided $|\beta| < 1$, which is required for convergence.

\begin{figure}[!ht]
    \centering
    \includegraphics[scale=0.7]{src/SerieChronologique/MA-Correlogram.png}
    \caption{Plots of the autocorrelation functions for two MA(3) processes. The autocorrelation for lag $k>q$ are all zero. $(a) \beta_1 = 0.7, \beta_2 = 0.5, \beta_3 = 0.2; (b) \beta_1 = -0.7, \beta_2 = 0.5, \beta_3 = -0.2$.}
\end{figure}

\subsection{Mized Models: The ARMA process}

\paragraph{Definition}
The ARMA model is a AR(p) + MA(q) models defined as
\[ x_t = \alpha_1 x_{t-1} + \alpha_2 x_{t-2} + ... + \alpha_p x_{t-p} + w_t + \beta_1 w_{t-1} + \beta_2 w_{t-2} + ... + \beta_q w_{t-q} \]
We can also express it as 
\[ \theta_p(\mathrm{B}) x_t = \phi_q(\mathrm{B}) w_t \]
\begin{itemize}
    \item[\textbullet] The process is stationary when the roots of $\theta$ are all exceed unity in absolute value.
    \item[\textbullet] The process is invertible when the roots of $\phi$ all exceed unity in absolute value.
    \item[\textbullet] The AR(p) model is the special case ARMA(p, 0).
    \item[\textbullet] The MA(q) model is the special case ARMA(0, q).
    \item[\textbullet] \textbf{Parameter parsimony.} When fitting to data, an ARMA model will often be more parameter efficient (i.e., require fewer parameters) than a single MA or AR model.
    \item[\textbullet] \textbf{Parameter redundancy.} When $\theta$ and $\phi$ share a common factor, a stationary model can be simplified. For example, the model:
        \begin{align*}
            (1 - \frac{1}{2}B)(1 - \frac{1}{3}B)x_t &= (1 - frac{1}{2}B)w_t \\
            (1 - \frac{1}{3}B)x_t &= w_t
        \end{align*}
\end{itemize}

\paragraph{Second-order Properties}
\begin{align*}
    \sigma^2 &= \sigma_w^2 \left( 1 + \frac{(\alpha+\beta)^2}{1 - \alpha^2} \right) \\
    \gamma_0 &= \sigma_w^2 \left( \frac{1 + 2 \alpha \beta + \beta^2}{1 - \alpha^2} \right) \\
    \gamma_k &= \sigma_w^2 (\alpha + \beta)\alpha^{k-1} \left( \frac{1 + \alpha \beta}{1 - \alpha^2} \right) \\
    \rho_k &= \frac{\alpha^{k-1}(\alpha+\beta)(1+\alpha\beta)}{1 + \alpha\beta + \beta^2} = \alpha \rho_{k-1}
\end{align*}

\section{Non-stationary Models}

\subsection{Differencing}
Differencing a serie $\{x_t\}$ can remove trends, whether these trend are stochastic, as in a random walk, or deterministic, as in the case of a linear trend. 

\subparagraph{Differencing random walk}
\[ \nabla x_t = x_t - x_{t-1} = w_t \]
which is a stationary white noise.

\subparagraph{Differencing linear trend}
\[ \nabla x_t = x_t - x_{t-1} = b + w_t - w_{t-1} \]
which is a stationary moving average process rather than white noise.

\paragraph{Integrated Model}
A serie $\{x_t\}$ is integrated of order d, $I(d)$, if the d\up{th} difference of $\{x_t\}$ is a white noise
\[ (1 - \mathrm{B})^d x_t = w_t \]

\begin{note}
    The random walk is a special case $I(1)$
\end{note}

\subsection{Non-Seasonal ARIMA Models}

\paragraph{Definition}
A time series $\{x_t\}$ follows an ARIMA($p, d, q$) process id the d\up{th} differences of the $\{x_t\}$ series are an ARIMA($p, q$) process
\[ \theta_p(\mathrm{B}(1 - \mathrm{B})^d x_t = \phi_q(\mathrm{B}) w_t \]

In general:
\begin{itemize}
    \item[\textbullet] $\text{ARIMA}(0,d,q) \equiv \text{IMA}(d, q)$
    \item[\textbullet] $\text{ARIMA}(p,d,0) \equiv \text{ARI}(p,d)$
\end{itemize}

\subsection{Seasonal ARIMA models}
A seasonal ARIMA model uses differencing at a lag equal to the number of seasons (s) to remove additive seasonal effects. As with lag 1 differencing to remove a trend, the lag s differencing introduces a moving average term.
\\
\\
The ARIMA$(p,d,q)(P,D,Q)_s$ model can be defined as
\[ \Theta_P(\mathrm{B}^s)\theta_p(\mathrm{ B})(1 - \mathrm{B}^s)^D (1 - \mathrm{B})^d x_t = \Phi_Q(\mathrm{B}^s) \phi_q(\mathrm{B}) w_t \]


\label{Chapter:SerieChronologique}

{ \color{red} Thissection resume the Chapters 5-9 of \emph{An Introduction to Generalized Linear Models}.}

\hl{Add to biblio:}

\section{Inference}

The two tool to do inference are \textbf{confidence intervals} and \textbf{hypothesis tests}. For GLM, a hypothesis tests can be use to compare two models, but their need to have the same probability function and the same link. Also, the null hypothesis $H_0$ is a sinpler model and must be a special case of the other. 

\subsection{Sampling distribution for the score statistic}
\paragraph{Score Function}
Let $\ell = \ln f(y)$ be the log-likehood function, then the score function $U$, is define as
\[ U_j = \frac{\partial \ell}{\partial \mu} = \sum_{i=1}^N \left[ \frac{(Y_i - \mu_i)}{\variance{Y_i}} x_{i,j} \left( \frac{\partial \mu_i}{\partial \eta} \right) \right] \]

\paragraph{Information matrix}
The information matrix is define as the variance-covariance matrix of the score function. This information matrix is defined as 
\[ I(\theta) = \variance{U} \]

\paragraph{Score Statistic}
If there is only one $\beta$, the score statistic has the asymptotic sampling distribution
\[ \frac{U}{\sqrt{I(\theta)}} \sim N(0, 1),  \frac{U^2}{I(\theta)} \sim \chi^2(1) \]

If there is a vector of $\underline{\beta}$
\[ U^T I(\theta)^{-1} U \sum \chi^2(p) \]

\section{Normal Linear Models}
\subsection{Basis Result}

\paragraph{Maximum likelihood}
The maximum likelihood estimation of $\beta$ is given by
\[ \mathrm{b} = (\mathrm{X}^T \mathrm{X})^{-1} \mathrm{X}^T \mathrm{y} \]
The estimator is unbiasedm with the variance-covariance matrix
\[ I(\theta)^{-1} = \sigma^2 (\mathrm{X}^T \mathrm{X}^{-1}) \]
However, the unbiased estimator of $\sigma^2$ is given by
\[ \hat{\sigma^2} = \frac{1}{N-p} (\mathrm{y} - \mathrm{X}\mathrm{b})^T \]

\paragraph{Least Square Estimation}
In the case of linear models, we obtain the same result as the maximum likelihood
\[ \mathrm{b} = (\mathrm{X}^T \mathrm{X})^{-1} \mathrm{X}^T \mathrm{y} \]

\paragraph{Deviance}
The diviance is define by the square of the error $\mathrm{\varepsilon}$.
\[ \frac{1}{\sigma^2} \mathrm{\varepsilon}^T \mathrm{\varepsilon} = \frac{1}{\sigma^2} (\mathrm{y} - \mathrm{X}\mathrm{b})^T(\mathrm{y} - \mathrm{X}\mathrm{b}) \]
or, in case of simple linear model,
\[ \frac{1}{\sigma^2} \sum (Y_i - \hat{Y_i})^2 \]


\label{Chapter:ExtendedLinearModels}


\chapter{Statistical Learning}
\hl{reference statistical learning (voir mas1)}

This section do not cover all the formulas talk in the book since we already seen it in \emph{Modèle Linéaire en actuariat}. This section talk more about the analysis of some statistical model.

\section{Statistical Learning}

\paragraph{Prediction}
\begin{align*}
    \esp{Y - \hat{Y}} &= \esp{f(x) + \varepsilon - \hat{f}(x)}^2 \\
                      &= [f(x) - \hat{f}(x)]^2 + \variance{\varepsilon} \\
                      &= (\text{Réductible}) + (\text{Irreductible)}
\end{align*}

\paragraph{Inference}
We are often interested in understanding the way that Y is affected as $X_1,...,X_p$ is changing. Inference mean that we want to understand the relationship between X and Y, or more specifically, to understand how Y changes as a function of $X_1,...,X_p$ 
\begin{itemize}
    \item Which predictors are associated with the response?
    \item What is the relationship between the response and each predictor?
    \item Can the relationship between Y and each predictor be adequately summarized using a linear equation, or is the relationship more compli cated?
\end{itemize}

\subsection{How Do We Estimate f?}
\paragraph{Parametric Methods}
Parametric methods involve a two-step model-based approach.

\begin{enumerate}
    \item  First, we make an assumption about the functional form, or shape, of f. For example, one very simple assumption is that f is linear in X: \[ f(X) = \beta_0 + \beta_1 X_1 + ... + \beta_p X_p \]
    \item  After a model has been selected, we need a procedure that uses the training data to fit or train the model.enumarate
\end{enumerate}

The potential disadvantage of a parametric approach is that the model we choose will usually not match the true unknown form of f. If the chosen model is too far from the true f, then our estimate will be poor. We can try to address this problem by choosing flexible models that can fit many different possible functional forms for f. But in general, fitting a more flexible model requires estimating a greater number of parameters. These more complex models can lead to a phenomenon known as overfitting the data.

\paragraph{Non-parametric Methods}
No assumption about the form of f is made. 
\begin{itemize}
    \item \textbf{Advantage:} Fit the model more closly to the data points.
    \item \textbf{Disadvantage:} Since they do not reduce the problem of estimating f to a
small number of parameters, a very large number of observations (far more
than is typically needed for a parametric approach) is required in order to
obtain an accurate estimate for f.
\end{itemize}

\begin{figure}[!ht]
\centering
\includegraphics[scale=0.7]{src/StatisticalLearning/Trade-Off_Prediction-Interpretability.PNG}
\caption{A representation of the tradeoff between flexibility and interpretability, using different statistical learning methods. In general, as the flexibility of a method increases, its interpretability decreases.}
\end{figure}

\subsection{Measuring the Quality of Fit}
In the regression setting, the most commonly-used measure is the mean squared error (MSE), given by
\[ \mathrm{MSE} = \frac{1}{n} \sumn (y_i - \hat{f}(x_i))^2  \]
The MSE will be small if the predicted responses are very close to the true responses.

Note that regardless of whether or not overfitting has
occurred, we almost always expect the training MSE to be smaller than
the test MSE because most statistical learning methods either directly or
indirectly seek to minimize the training MSE.

\paragraph{Overfitting}
Overfitting refers specifically
to the case in which a less flexible model would have yielded a smaller
test MSE.

\paragraph{The Bias-Variance Trade-Off}
The equation below tells us that in order to minimize the expected test error,
we need to select a statistical learning method that simultaneously achieves
low variance and low bias. Note that variance is inherently a nonnegative
quantity, and squared bias is also nonnegative. Hence, we see that the
expected test MSE can never lie below Var(?), the irreducible error
\[ \esp{y_0 - \hat{f}(x_0)} \variance{\hat{f(x_0)}} + [ \mathrm{Bias}(\hat{f}(x_0))]^2 + \variance{\varepsilon} \]
\begin{itemize}
    \item \textbf{Variance} refers to the amount by which $\hat{f}$ would change if we estimated it using a different training data set.  In general, more flexible statistical methods have higher variance.
    \item \textbf{Bias} refers to the error that is introduced by approximating a real-life problem, which may be extremely complicated, by a much simpler model. Generally, more flexible methods result in less bias.
\end{itemize}

\begin{figure}[!ht]
    \centering
    \includegraphics[scale=0.7]{src/StatisticalLearning/Trade-off_Variance-Bias.PNG}
    \caption{Bias-Variance trade-Off}
\end{figure}

\section{Linear Regression}

\subsection{Simple Linear Regression}
\paragraph{Definition}
\[ \hat{y} = \hatbeta_0 + \hatbeta_1 x \]
where the $\hatbeta$ are estimate using the \textbf{least squares} criterion.

\begin{figure}[!ht]
    \centering
    \includegraphics[scale=0.5]{src/StatisticalLearning/SimpleLinearRegression.PNG}
    \caption{Exemple of simple linear regression}
\end{figure}

\paragraph{Residual sum od squares}
We define the residual sum of square as
\[ \mathrm{RSS} = \mathrm{SSE} = \sumn \varepsilon_i = \sumn (y_i - \hat{y}_i)^2 \]

\paragraph{$R^2$ Statistic} 
For simple linear regression, $R^2 = \covar{X, Y}$.

Backward selection cannot be used if p > n, while forward selection can
always be used. Forward selection is a greedy approach, and might include
variables early that later become redundant. Mixed selection can remedy
this.

\section{Multiple Linear Regression}

\subsection{Potential Problems}
When we fit a linear regression model to a particular data set, many prob-
lems may occur. Most common among these are the following:

\paragraph{1. Non-linearity of the Data}
The linear regression model assumes that there is a straight-line relationship between the predictors and the response. If the true relationship is far from linear, then virtually all of the conclusions that we draw from the fit are suspect. In addition, the prediction accuracy of the model can be significantly reduced.

Residual plots are a useful graphical tool for identifying non-linearity. If the residual plot indicates that there are non-linear associations in the data, then a simple approach is to use non-linear transformations of the predictors, such as $\ln(X)$, $\sqrt{X}$, and $X^2$ , in the regression model.

\paragraph{2. Correlation of Error Terms}
An important assumption of the linear regression model is that the error terms, $\varepsilon_1,...,\varepsilon_n$ are uncorrelated. If in fact there is correlation among the error terms, then the estimated standard errors will tend to underestimate the true standard errors.

\begin{figure}[!ht]
    \centering
    \includegraphics[scale=0.6]{src/StatisticalLearning/CorrelationInErrorTerms.PNG}
    \caption{Plots of residuals time series data sets with differing levels of correlation $\rho$ between error terms for adjacent time points. The graphs on top is good for linear regression}
\end{figure}

\paragraph{3. Non-constant Variance of Error Terms}
Another important assumption of the linear regression model is that the
error terms have a constant variance, $\variance{\varepsilon} = \sigma^2$. If not, we can recognize \textbf{heteroscedasticity} with a funnel shape in the residual plot.

\begin{figure}[!ht]
    \centering
    \includegraphics[scale=0.6]{src/StatisticalLearning/Heteroscedasticity.PNG}
    \caption{Residual plots. Left: The funnel shape indicates heteroscedasticity. Right: The predictor has been log-transformed, and there is now no evidence of heteroscedasticity.}
\end{figure}

 When faced with this problem, one possible solution is to transform the response Y using a concave function such as $\ln(Y)$ or $\sqrt{Y}$. Such a transformation results in a greater amount of shrinkage of the larger responses, leading to a reduction in heteroscedasticity. Sometimes, we can also use \textbf{weighted least squares}.

\paragraph{4. Outliers}
An outlier is a point for which $y_i$ is far from the value predicted by the model. Outliers can arise for a variety of reasons, such as incorrect recording of an observation during data collection.

\begin{figure}[!ht]
    \centering
    \includegraphics[scale=0.6]{src/StatisticalLearning/Outlier.PNG}
    \caption{Center: The residual plot clearly identifies the outlier. Right: The outlier has a studentized residual of 6; typically we expect values between -3 and 3.}
\end{figure}

Outlier can affect our MSE estimation, resolving in inadequate $R^2$ or confidence interval. We can remove the outlier to resolve this issue.

\paragraph{5. High Leverage Points}
In constrast of \emph{outlier} that are unusual $y_i$, leverage are unusual value of $x_i$. 

\begin{figure}[!ht]
    \centering
    \includegraphics[scale=0.6]{src/StatisticalLearning/Leverage.PNG}
    \caption{Left: Observation 41 is a high leverage point, while 20 is not. The red line is the fit to all the data, and the blue line is the fit with observation 41 removed. Center: The red observation is not unusual in terms of its $X_1$ value or its $X_2$ value, but still falls outside the bulk of the data, and hence has high leverage. Right: Observation 41 has a high leverage and a high residual.}
\end{figure}

For simple linear regression, we can compute the leverage statistic define as
\[ h_i = \frac{1}{n} + \frac{(x_i - \bar{x})^2}{\sum (x_i - \bar{x})^2} \]
where $\frac{1}{n} < h_i < 1$ and $\sum h_i = \frac{(p+1)}{n}$.
So if a given observation has a leverage statistic that greatly exceeds $(p+1)/n$, hen we may suspect that the corresponding point has high leverage.

\paragraph{6. Collinearity}
Collinearity refers to the situation in which two or more predictor variables collinearity are closely related to one another.

\begin{figure}[!ht]
    \centering
    \includegraphics[scale=0.6]{src/StatisticalLearning/Collinearity.PNG}
    \caption{Left: A plot of age versus limit. These two variables are not collinear. Right: A plot of rating versus limit. There is high collinearity.}
\end{figure}

The presence of collinearity can pose problems in the regression context, since it can be difficult to separate out the individual effects of collinear variables on the response.


\label{Chapter:StatisticalLearning}

% ANNEXES DU DOCUMENT
\appendix
\chapter{Principales distribution de probabilité utilisées}
introduction

\include{Appendix-PreuveImportante}
\label{Appendix:PreuveImportante}

\include{Appendix-GitWorkflow}
\label{Appendix:GitWorkflow}



\end{document}
